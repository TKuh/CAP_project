\CapPkg supports categorical constructions 
for a category $\mathbf{C}$
of the following types:

\begin{itemize}
 \item Basic constructions which only depend on the fact that $\mathbf{C}$ is category,
 such as composition ($\PreCompose$), identity ($\IdentityMorphism$), decisions whether
 a morphism is a mono, epi, or iso ($\IsMonomorphism, \IsEpimorphism, \IsIsomorphism$).
 \item Some limit and colimit constructions, such as direct product ($\DirectProduct$),
 coproduct ($\Coproduct$), fiber product ($\FiberProduct$), pushout ($\Pushout$),
 kernel ($\KernelObject$), cokernel ($\Cokernel$).
 \item Constructions for monoidal categories, such as tensor product ($\TensorProductOnObjects$, $\TensorProductOnMorphisms$),
 unitors ($\LeftUnitor$, $\RightUnitor$), associators \\($\AssociatorLeftToRight$, $\AssociatorRightToLeft$),
 internal hom ($\InternalHomOnObjects$, $\InternalHomOnMorphisms$).
\end{itemize}

\begin{notation}
 For the definition of a category, see chapter \ref{chapter:specifications}, section \ref{section:categories}.
\end{notation}


\section{Basic Constructions}

\section{Limits and Colimits}\label{section:universalobjects}

In \CapPkg, support is given for the implementation of some special kinds of limits and
colimits. We first give the definition of a limit and a colimit.
Let $\mathbf{C}$ be a category. Let $\mathbf{I}$ be another category (called an \textbf{index category}) and $D: \mathbf{I} \rightarrow \mathbf{C}$
be a functor (called a \textbf{diagram}). 
\begin{definition} 
 A \textbf{source} of $D$ is a collection of morphisms $( s_i: S \rightarrow D_i)_{i \in \mathbf{I}}$ such that
 $\left( D( i \rightarrow j ) \circ s_i \right) \sim_{S,D_j} s_j$ for every arrow $(i \rightarrow j) \in \mathbf{I}$.
\end{definition}
 
\begin{definition}
 A triple $(L, \lambda, u)$ consisting of the following data:
 \begin{enumerate}
  \item an object $L \in \mathbf{C}$,
  \item a source $\lambda = ( \lambda_i: L \rightarrow D_i)_{i \in \mathbf{I}}$,
  \item a dependent function $u$ mapping every source $\tau = ( \tau_i: T \rightarrow D_i )_{i \in \mathbf{I}}$
        to a morphism $u( \tau ): T \rightarrow L$ such that
        $\lambda_i \circ u( \tau ) \sim_{T,D_i} \tau_i$,
 \end{enumerate}
 is called a \textbf{limit of the diagram $D$}, if the morphisms $u(\tau)$ are uniquely determined up to
 congruence of morphisms.
\end{definition}

\begin{definition}
 A \textbf{sink} of $D$ is a collection of morphisms $( s_i: D_i \rightarrow S )_{i \in \mathbf{I}}$
 such that $\left(s_i \circ D( i \rightarrow j )\right) \sim_{D_i, S} s_j$ for every arrow $(i \rightarrow j) \in \mathbf{I}$.
\end{definition}


\begin{definition}
 A triple $(C, c, u)$ consisting of the following data:
 \begin{enumerate}
  \item an object $C \in \mathbf{C}$,
  \item a sink $c = ( c_i: D_i \rightarrow C )_{i \in \mathbf{I}}$,
  \item a dependent function $u$ mapping every source $\tau = ( \tau_i: D_i \rightarrow T )_{i \in \mathbf{I}}$
        to a morphism $u( \tau ): C \rightarrow T$ such that
        $u(\tau) \circ c_{i} \sim_{D_i,T} \tau_i$,
 \end{enumerate}
 is called a \textbf{colimit of the diagram $D$}, if the morphisms $u(\tau)$ are uniquely determined up to
 congruence of morphisms.
\end{definition}

\begin{example}
 Examples of limits are kernels, direct products, fiber products.
\end{example}

\begin{example}
 Examples of colimits are cokernels, coproducts, pushouts.
\end{example}

\subsection{Limits in \CapPkg}

For every limit $\left(L, \lambda = ( \lambda_i: L \rightarrow D_i)_{i \in \mathbf{I}} \right)$ in \CapPkg, 
there are three basic operations.

\begin{enumerate}
 \item A basic operation returning the limit object. The argument is the diagram:
 \[
  D \mapsto L.
 \]
 \item A basic operation returning the limit source. The argument is the diagram:
 \[
  D \mapsto \lambda.
 \]
 \item A basic operation returning the morphism given by the universal property. 
       The arguments are the diagram and a test source:
 \[
  ( D, \tau ) \mapsto u(\tau).
 \]
\end{enumerate}

\begin{example}
 In the case of a kernel, the three basic operations in question are
 \begin{enumerate}
  \item $\KernelObject$,
  \item $\KernelEmb$,
  \item $\KernelLift$.
 \end{enumerate} 
\end{example}

In addition, there are two more basic operations which are variants of the second
and the third basic operation one.

\begin{enumerate}
 \item A basic operation returning the limit source. The arguments are the diagram
       and an object equal ($\IsEqualForObjects$) to the limit object:
       \[
        ( D, L ) \mapsto \lambda.
       \]
  \item A basic operation returning the morphism given by the universal property.
        The arguments are the diagram, a test source, and an object equal ($\IsEqualForObjects$)
        to the limit object:
        \[
         ( D, \tau, L ) \mapsto u( \tau ).
        \]
\end{enumerate}

\begin{example}
 In the case of a kernel, the two variants in question are
 \begin{enumerate}
  \item $\KernelEmbWithGivenKernelObject$,
  \item $\KernelLiftWithGivenKernelObject$.
 \end{enumerate} 
\end{example}

We call such basic operations \texttt{WithGiven} operations.
We recommend not to call the \texttt{WithGiven} operations but only to implement them
as helpers in the following sense:
Depending on which of the above five basic operations are implemented, \CapPkg
automatically tries to fill in the missing operations:

\begin{itemize}
 \item Given $D \mapsto L$ and $( D, L ) \mapsto \lambda$, \CapPkg derives $D \mapsto \lambda$.
 \item Given $D \mapsto L$ and $( D, \tau, L ) \mapsto u( \tau )$, \CapPkg derives $( D, \tau ) \mapsto u(\tau)$.
\end{itemize}

Moreover, if the three basic operations $D \mapsto L$, $D \mapsto \lambda$, and $( D, L ) \mapsto \lambda$
are all given, then \CapPkg automatically enhances $D \mapsto \lambda$ with a \textbf{redirect function}.
The redirect function first looks if the limit object $L$ is already in the cache of $D \mapsto L$
(compared with a user given equality function, default option is $\IsEqualForObjects$).
If this is the case, \CapPkg automatically calls the \texttt{WithGiven} operation $( D, L ) \mapsto \lambda$
with the input $L$ found in the cache.

Analogously, \CapPkg enhances $( D, \tau ) \mapsto u(\tau)$ with a redirect function if
the three basic operations $D \mapsto L, ( D, \tau ) \mapsto u(\tau), ( D, \tau, L ) \mapsto u( \tau )$ are all given.

We will now define the limits and colimits available in \CapPkg.

\subsection{Direct Product}

\begin{definition}
 For a given list of objects $D = ( P_1, \dots, P_n )$, a direct product of $D$ consists of three parts:
 \begin{enumerate}
  \item an object $P$,
  \item a list of morphisms $\pi = ( \pi_i: P \rightarrow P_i )_{i = 1 \dots n}$ 
  \item a dependent function $u$ mapping each list of morphisms $\tau = ( \tau_i: T \rightarrow P_i )_{i = 1, \dots, n}$ 
  to a morphism $u(\tau): T \rightarrow P$ such that $\pi_i \circ u( \tau ) \sim_{T,P_i} \tau_i$ for all $i = 1, \dots, n$.
 \end{enumerate}
 The triple $( P, \pi, u )$ is called a \textbf{direct product of $D$} if the morphisms $u( \tau )$ are uniquely determined up to
 congruence of morphisms.
\end{definition}

\subsection{Coproduct}

\begin{definition}
 For a given list of objects $D = ( I_1, \dots, I_n )$, a coproduct of $D$ consists of three parts:
 \begin{enumerate}
  \item an object $I$,
  \item a list of morphisms $\iota = ( \iota_i: I_i \rightarrow I )_{i = 1 \dots n}$
  \item a dependent function $u$ mapping each list of morphisms $\tau = ( \tau_i: I_i \rightarrow T )$
  to a morphism $u( \tau ): I \rightarrow T$ such that $u( \tau ) \circ \iota_i \sim_{I_i, T} \tau_i$ for all $i = 1, \dots, n$.
 \end{enumerate}
 The triple $( I, \iota, u )$ is called a \textbf{coproduct of $D$} if the morphisms $u( \tau )$ are uniquely determined up to
 congruence of morphisms.
\end{definition}

\subsection{Fiber Product}

\begin{definition}
 For a given list of morphisms $D = ( \beta_i: P_i \rightarrow B )_{i = 1 \dots n}$, 
 a fiber product of $D$ consists of three parts:
 \begin{enumerate}
  \item an object $P$,
  \item a list of morphisms $\pi = ( \pi_i: P \rightarrow P_i )_{i = 1 \dots n}$ such that
        $\beta_i \circ \pi_i  \sim_{P, B} \beta_j \circ \pi_j$ for all pairs $i,j$.
  \item a dependent function $u$ mapping each list of morphisms
        $\tau = ( \tau_i: T \rightarrow P_i )$ such that
        $\beta_i \circ \tau_i  \sim_{T, B} \beta_j \circ \tau_j$ for all pairs $i,j$
        to a morphism $u( \tau ): T \rightarrow P$ such that
        $\pi_i \circ u( \tau ) \sim_{T, P_i} \tau_i$ for all $i = 1, \dots, n$.
 \end{enumerate}
 The triple $( P, \pi, u )$ is called a \textbf{fiber product of $D$} if the morphisms $u( \tau )$ are uniquely determined up to
 congruence of morphisms.
\end{definition}

\subsection{Pushout}

\begin{definition}
 For a given list of morphisms $D = ( \beta_i: B \rightarrow I_i )_{i = 1 \dots n}$,
 a pushout of $D$ consists of three parts:
 \begin{enumerate}
  \item an object $I$,
  \item a list of morphisms $\iota = ( \iota_i: I_i \rightarrow I )_{i = 1 \dots n}$ such that
        $\iota_i \circ \beta_i \sim_{B,I} \iota_j \circ \beta_j$ for all pairs $i,j$,
  \item a dependent function $u$ mapping each list of morphisms
        $\tau = ( \tau_i: I_i \rightarrow T )_{i = 1 \dots n}$ such that
        $\tau_i \circ \beta_i \sim_{B,T} \tau_j \circ \beta_j$
        to a morphism $u( \tau ): I \rightarrow T$ such that
        $u( \tau ) \circ \iota_i \sim_{I_i, T} \tau_i$ for all $i = 1, \dots, n$.
 \end{enumerate}
  The triple $( I, \iota, u )$ is called a \textbf{pushout of $D$} if the morphisms $u( \tau )$ are uniquely determined up to
  congruence of morphisms.
\end{definition}

\subsection{Kernel}

\begin{definition}
  For a given morphism $\alpha: A \rightarrow B$, a kernel of $\alpha$ consists of three parts:
 \begin{enumerate}
  \item an object $K$,
  \item a morphism $\iota: K \rightarrow A$ such that $\alpha \circ \iota \sim_{K,B} 0$,
  \item a dependent function $u$ mapping each morphism $\tau: T \rightarrow A$ satisfying $\alpha \circ \tau \sim_{T,B} 0$ to
  a morphism $u(\tau): T \rightarrow K$ such that $\iota \circ u( \tau ) \sim_{T,A} \tau$. 
 \end{enumerate}
 The triple $( K, \iota, u )$ is called a \textbf{kernel of $\alpha$} if the morphisms $u( \tau )$ are uniquely determined up to
 congruence of morphisms. The situation can be depicted as follows:
 
 \begin{center}
\begin{tikzpicture}[label/.style={postaction={
      decorate,
      decoration={markings, mark=at position .5 with \node #1;}}}]
      \coordinate (r) at (1.5,0);
      \coordinate (u) at (0,0.75);
      
      \node (M) {$A$};
      \node (N) at ($(M)+(r)$) {$B$};
      \node (K) at ($(M)-(r)+(u)$) {$K$};
      \node (L) at ($(K)-2*(u)$) {$T$};
      \draw[->,thick] (M) -- node[above]{$\alpha$} (N);
      \draw[->, thick] (K) to [bend left] node[above] {$0$}(N);
      \draw[right hook->,thick] (K) -- node[above]{$\iota$} (M);
      \draw[->,thick] (L) -- node[above]{$\tau$} (M);
      \draw[->, thick] (L) to [bend right] node[below] {$0$}(N);
      \draw[->,dotted,thick] (L) -- node[left]{$u( \tau )$} (K);
\end{tikzpicture}
\end{center}
\end{definition}

\subsection{Cokernel}

\begin{definition}
 For a given morphism $\alpha: A \rightarrow B$, a cokernel of $\alpha$ consists of three parts:
 \begin{enumerate}
  \item an object $K$,
  \item a morphism $\epsilon: B \rightarrow K$ such that $\epsilon \circ \alpha \sim_{A,K} 0$,
  \item a dependent function $u$ mapping each $\tau: B \rightarrow T$ satisfying $\tau \circ \alpha \sim_{A, T} 0$
        to a morphism $u(\tau):K \rightarrow T$ such that $u(\tau) \circ \epsilon \sim_{B,T} \tau$.
 \end{enumerate}
  The triple $( K, \epsilon, u )$ is called a \textbf{cokernel of $\alpha$} if the morphisms $u( \tau )$ are uniquely determined up to
 congruence of morphisms.
\end{definition}

\subsection{Terminal Object}

\begin{definition}
 A terminal object consists of two parts:
 \begin{enumerate}
  \item an object $T$,
  \item a function $u$ mapping each object $A$ to a morphism $u( A ): A \rightarrow T$.
 \end{enumerate}
 The pair $( T, u )$ is called a \textbf{terminal object} if the morphisms $u( A )$ are uniquely determined up to
 congruence of morphisms.
\end{definition}

\begin{remark}
 The correspondingding diagram $D$ is given by the unique functor $\empty \rightarrow \mathbf{A}$.
 Thus the source of a terminal object can be omited.
\end{remark}

\subsection{Initial Object}

\begin{definition}
 An initial object consists of two parts:
 \begin{enumerate}
  \item an object $I$,
  \item a function $u$ mapping each object $A$ to a morphism $u( A ): I \rightarrow A$.
 \end{enumerate}
  The pair $(I,u)$ is called a \textbf{initial object} if the morphisms $u(A)$ are uniquely determined up to
 congruence of morphisms.
\end{definition}

\subsection{Zero Object}

Roughly speaking, a zero object is both an inital and a terminal object.
Of course, the constructive definition of a zero object has to include
algorithms (or witnesses) of being initial and being terminal.

\begin{definition}
  A zero object consists of three parts:
  \begin{enumerate}
   \item an object $Z$,
   \item a function $u_{\mathrm{in}}$ mapping each object $A$ to a morphism $u_{\mathrm{in}}(A): A \rightarrow Z$,
   \item a function $u_{\mathrm{out}}$ mapping each object $A$ to a morphism $u_{\mathrm{out}}(A): Z \rightarrow A$.
  \end{enumerate}
  The triple $(Z, u_{\mathrm{in}}, u_{\mathrm{out}})$ is called a \textbf{zero object} if the morphisms 
  $u_{\mathrm{in}}(A)$, $u_{\mathrm{out}}(A)$ are uniquely determined up to congruence of morphisms.
\end{definition}


\subsection{Direct Sum}

Roughly speaking, a direct sum is an object in an additive category which is both a direct product and
a coproduct such that the injections and projections are compatible.
We now state the constructive definition.

\begin{definition}\label{definition:direct_sum}
 For a given list $D = (S_1, \dots, S_n)$, a direct sum consists of five parts:
 \begin{enumerate}
  \item an object $S$,
  \item a list of morphisms $\pi = (\pi_i: S \rightarrow S_i)_{i = 1 \dots n}$,
  \item a list of morphisms $\iota = (\iota_i: S_i \rightarrow S)_{i = 1 \dots n}$,
  \item a dependent function $u_{\mathrm{in}}$ mapping every list $\tau = ( \tau_i: T \rightarrow S_i )_{i = 1 \dots n}$
        to a morphism $u_{\mathrm{in}}(\tau): T \rightarrow S$ such that
        $\pi_i \circ u_{\mathrm{in}}(\tau) \sim_{T,S_i} \tau_i$ for all $i = 1, \dots, n$.
  \item a dependent function $u_{\mathrm{out}}$ mapping every list $\tau = ( \tau_i: S_i \rightarrow T )_{i = 1 \dots n}$
        to a morphism $u_{\mathrm{out}}(\tau): S \rightarrow T$ such that
        $u_{\mathrm{out}}(\tau) \circ \iota_i \sim_{S_i, T} \tau_i$ for all $i = 1, \dots, n$,
 \end{enumerate}
 such that
 \begin{enumerate}
  \item $\sum_{i=1}^{n} \iota_i \circ \pi_i = \id_S$,
  \item $\pi_j \circ \iota_i = \delta_{i,j}$,
 \end{enumerate}
  where $\delta_{i,j} \in \Hom( S_i, S_j )$ is the identity if $i=j$, and $0$ otherwise.
  The $5$-tuple $(S, \pi, \iota, u_{\mathrm{in}}, u_{\mathrm{out}})$ is called a \textbf{direct sum of $D$}
  if the morphisms 
  $u_{\mathrm{in}}(\tau)$, $u_{\mathrm{out}}(\tau)$ are uniquely determined up to congruence of morphisms.
\end{definition}

\begin{remark}
 Given a triple $( S, \pi, \iota )$ satisfying the conditions of definition \ref{definition:direct_sum},
 we can construct $u_{\mathrm{in}}$, $u_{\mathrm{out}}$ in unique way up to congruence of morphisms
 such that $( S, \pi, \iota, u_{\mathrm{in}}, u_{\mathrm{out}} )$ is a direct sum.
\end{remark}


\subsection{Image}

\begin{definition}
 For a given morphism $\alpha: A \rightarrow B$, an image of $\alpha$ consists of four parts:
 \begin{enumerate}
  \item an object $I$,
  \item a morphism $c: A \rightarrow I$,
  \item a monomorphism $\iota: I \hookrightarrow B$ such that $\iota \circ c \sim_{A,B} \alpha$,
  \item a dependent function $u$ mapping each pair of morphisms $\tau = ( \tau_1: A \rightarrow T, \tau_2: T \hookrightarrow B )$
  where $\tau_2$ is a monomorphism
  such that $\tau_2 \circ \tau_1 \sim_{A,B} \alpha$ to a morphism
  $u(\tau): I \rightarrow T$ such that
  $\tau_2 \circ u(\tau) \sim_{I,B} \iota$ and $u(\tau) \circ c \sim_{A,T} \tau_1$.
 \end{enumerate}
  The $4$-tuple $( I, c, \iota, u )$ is called an \textbf{image of $\alpha$} if the morphisms $u( \tau )$ are uniquely determined up to
 congruence of morphisms.
\end{definition}

\subsection{Coimage}

\begin{definition}
 For a given morphism $\alpha: A \rightarrow B$, a coimage of $\alpha$ consists of four parts:
 \begin{enumerate}
  \item an object $C$,
  \item an epimorphism $\pi: A \twoheadrightarrow C$,
  \item a morphism $a: C \rightarrow B$ such that $a \circ \pi \sim_{A,B} \alpha$,
  \item a dependent function $u$ mapping each pair of morphisms $\tau = ( \tau_1: A \twoheadrightarrow T, \tau_2: T \rightarrow B )$
  where $\tau_1$ is an epimorphism
  such that $\tau_2 \circ \tau_1 \sim_{A,B} \alpha$ to a morphism
  $u(\tau): T \rightarrow C$ such that
  $u( \tau ) \circ \tau_1 \sim_{A,C} \pi$ and $a \circ u( \tau ) \sim_{T,B} \tau_2$.
 \end{enumerate}
  The $4$-tuple $( C, \pi, a, u )$ is called a \textbf{coimage of $\alpha$} if the morphisms $u( \tau )$ are uniquely determined up to
 congruence of morphisms.
\end{definition}



\section{Categories with more Structure}

\CapPkg supports various notions of categories with additional structure.
If we want to implement such categories, we set the corresponding
\GAP properties to true right after the call of the constructor
\texttt{CreateCapCategory}.

\subsection{Categories Enriched over Commutative Regular Semigroups}

\begin{documentation}
 The corresponding \GAP property is given by
 \begin{center}
  \texttt{IsEnrichedOverCommutativeRegularSemigroup}.  
 \end{center}
\end{documentation}


\subsection{Ab-Categories}

\begin{documentation}
 The corresponding \GAP property is given by
 \begin{center}
  \texttt{IsAbCategory}.  
 \end{center}
\end{documentation}

\begin{definition}[\cite{MLCWM}]
 A category $\mathbf{C}$ enriched over abelian groups is called an \textbf{$\mathbf{Ab}$-category}, i.e.,
 every set of homomorphisms is equipped with a structure of an abelian group
 such that composition is bilinear.
\end{definition}

Basic operations for $\mathbf{Ab}$-categories coming from the existential quantifiers
in the definition.
\begin{itemize}
 \item \texttt{ZeroMorphism}
 \item \texttt{IsZeroForMorphisms}
 \item \texttt{AdditionForMorphisms}
 \item \texttt{AdditiveInverseForMorphisms}
\end{itemize}

\subsection{Additive Categories}

\begin{documentation}
 The corresponding \GAP property is given by
 \begin{center}
  \texttt{IsAdditiveCategory}.  
 \end{center}
\end{documentation}

\begin{definition}[\cite{MLCWM}]
 An $\mathbf{Ab}$-category $\mathbf{C}$ is called \textbf{additive}
 if it has a zero object and direct sums for all pairs of objects.
\end{definition}

Basic operations for additive categories coming from the existential quantifiers
in the definition
(in addition to the basic operations for $\mathbf{Ab}$-categories):
\begin{itemize}
 \item \texttt{ZeroObject}
 \item \texttt{UniversalMorphismFromZeroObject}
 \item \texttt{UniversalMorphismIntoZeroObject}
 \item \texttt{DirectSum}
 \item \texttt{ProjectionInFactorOfDirectSum}
 \item \texttt{InjectionOfCofactorOfDirectSum}
 \item \texttt{UniversalMorphismIntoDirectSum}
 \item \texttt{UniversalMorphismFromDirectSum}
\end{itemize}


\subsection{Preabelian Categories}

\begin{documentation}
 The corresponding \GAP property is given by
 \begin{center}
  \texttt{IsPreAbelianCategory}.  
 \end{center}
\end{documentation}

\begin{definition}
 An additive category $\mathbf{C}$ is called \textbf{preabelian} 
 if it has kernels and cokernels for every morphism.
\end{definition}

Basic operations for preadditive categories coming from the existential quantifiers
in the definition
(in addition to the basic operations for additive categories):
\begin{itemize}
 \item \texttt{KernelObject}
 \item \texttt{KernelEmb}
 \item \texttt{KernelLift}
 \item \texttt{Cokernel}
 \item \texttt{CokernelProj}
 \item \texttt{CokernelColift}
\end{itemize}


\subsection{Abelian Categories}[\cite{MLCWM}]

\begin{documentation}
 The corresponding \GAP property is given by
 \begin{center}
  \texttt{IsAbelianCategory}.  
 \end{center}
\end{documentation}

\begin{definition}
 A preabelian category $\mathbf{C}$ is called \textbf{abelian}
 if every monomorphism can be regarded as a kernel embedding
 and every epimorphism can be regarded cokernel projection.
\end{definition}

Basic operations for abelian categories coming from the existential quantifiers
in the definition
(in addition to the basic operations for preabelian categories):
\begin{itemize}
 \item \texttt{MonoAsKernelLift}
 \item \texttt{EpiAsCokernelColift}
\end{itemize}

\subsection{Monoidal Categories}

\begin{documentation}
 The corresponding \GAP property is given by
 \begin{center}
  \texttt{IsMonoidalCategory}.  
 \end{center}
\end{documentation}

\begin{definition}[\cite{MLCWM}]
 A $6$-tuple $( \mathbf{C}, \otimes, 1, \alpha, \lambda, \rho )$
 consisting of 
 \begin{itemize}
  \item a category $\mathbf{C}$, 
  \item a functor $\otimes: \mathbf{C} \times \mathbf{C} \rightarrow \mathbf{C}$,
  \item an object $1 \in \mathbf{C}$, 
  \item a natural isomorphism $\alpha_{a,b,c}: a \otimes (b \otimes c) \cong (a \otimes b) \otimes c$,
  \item a natural isomorphism $\lambda_{a}: 1 \otimes a \cong a$, 
  \item a natural isomorphism $\rho_{a}: a \otimes 1 \cong a$,
 \end{itemize}
 is called a \textbf{monoidal category} if the pentagonal diagram
 \begin{center} 
 \begin{tikzpicture}[commutative diagrams/every diagram]
\node (P0) at (90:2.8cm) {$ a \otimes ( b \otimes ( c \otimes d ) )$};
\node (P1) at (90+72:2.5cm) {$a \otimes ( ( b \otimes c ) \otimes d )$} ;
\node (P2) at (90+2*72:2.5cm) {\makebox[5ex][r]{$( a \otimes ( b \otimes c ) ) \otimes d$}};
\node (P3) at (90+3*72:2.5cm) {\makebox[5ex][l]{$((a \otimes b) \otimes c) \otimes d$}};
\node (P4) at (90+4*72:2.5cm) {$( a \otimes b ) \otimes ( c \otimes d )$};
\path[commutative diagrams/.cd, every arrow, every label]
(P0) edge node[swap] {$\id \otimes \alpha_{b,c,d}$} (P1)
(P1) edge node[swap] {$\alpha_{a, b \otimes c, d}$} (P2)
(P2) edge node {$\alpha_{a,b,c} \otimes \id$} (P3)
(P4) edge node {$\alpha_{a \otimes b, c, d}$} (P3)
(P0) edge node {$\alpha_{a,b,c \otimes d}$} (P4);
\end{tikzpicture}
\end{center}

and the triangular diagram

 \begin{center}
  \begin{tikzcd}
  a \otimes (1 \otimes c) \arrow{r}{\alpha_{a,1,c}} \arrow{dr}[description]{\id \otimes \lambda_{c}}
  &(a \otimes 1) \otimes c \arrow{d}[right]{\rho_{a} \otimes \id} \\
  &a \otimes c
  \end{tikzcd}
 \end{center}

commute.
\end{definition}

Basic operations for monoidal categories coming from the existential quantifiers
in the definition:
\begin{itemize}
 \item \texttt{TensorProductOnObjects}
 \item \texttt{TensorProductOnMorphisms}
 \item \texttt{TensorUnit}
 \item \texttt{AssociatorLeftToRight}
 \item \texttt{AssociatorRightToLeft}
 \item \texttt{LeftUnitor}
 \item \texttt{LeftUnitorInverse}
 \item \texttt{RightUnitor}
 \item \texttt{RightUnitorInverse}
 
\end{itemize}


\subsection{Braided Monoidal Categories}

\begin{documentation}
 The corresponding \GAP property is given by
 \begin{center}
  \texttt{IsBraidedMonoidalCategory}.  
 \end{center}
\end{documentation}

\begin{definition}[\cite{MLCWM}]
 A monoidal category $\mathbf{C}$ equipped with a natural isomorphism
 \[
  B_{a,b}: a \otimes b \cong b \otimes a
 \]
 is called a \textbf{braided monoidal category}
 if the following diagrams commute:
 
 \begin{itemize}
  \item Compatibility with the tensor unit:
  \begin{center}
    \begin{tikzcd}
    a \otimes 1 \arrow{r}{B_{a,1}} \arrow{dr}[left]{\rho_a} & 1 \otimes a \arrow{d}{\lambda_a} \\
    & a
    \end{tikzcd}
  \end{center}
 
  \item Swapping $b,c$ in the term $(a \otimes b) \otimes c$:
  \begin{center}
    \begin{tikzcd}
    (a \otimes b) \otimes c \arrow{r}{B} \arrow{d}{\alpha^{-1}} & c \otimes ( a \otimes b ) \arrow{d}{\alpha} \\
    a \otimes ( b \otimes c ) \arrow{d}{\id \otimes B} & ( c \otimes a ) \otimes b \arrow{d}{B \otimes \id} \\
    a \otimes ( c \otimes b ) \arrow{r}{\alpha} & ( a \otimes c ) \otimes b
    \end{tikzcd}
  \end{center}
  
  \item Swapping $a,b$ in the term $a \otimes (b \otimes c)$:
  \begin{center}
    \begin{tikzcd}
    a \otimes ( b \otimes c ) \arrow{r}{B} \arrow{d}{\alpha} & ( b \otimes c ) \otimes a \arrow{d}{\alpha^{-1}} \\
    ( a \otimes b ) \otimes c \arrow{d}{B \otimes \id} & b \otimes ( c \otimes a ) \arrow{d}{\id \otimes B} \\
    ( b \otimes a ) \otimes c \arrow{r}{\alpha} & b \otimes ( a \otimes c )
    \end{tikzcd}
  \end{center}
 \end{itemize}
\end{definition}

Basic operations for braided monoidal categories coming from the existential quantifiers
in the definition
(in addition to the basic operations for monoidal categories):
\begin{itemize}
 \item \texttt{Braiding}
 \item \texttt{BraidingInverse}
\end{itemize}

\subsection{Symmetric Monoidal Categories}

\begin{documentation}
 The corresponding \GAP property is given by
 \begin{center}
  \texttt{IsSymmetricMonoidalCategory}.  
 \end{center}
\end{documentation}

\begin{definition}[\cite{MLCWM}]
 A braided monoidal category $\mathbf{C}$ is called \textbf{symmetric monoidal category}
 if $B_{a,b}^{-1} = B_{b,a}$.
\end{definition}


\subsection{Symmetric Closed Monoidal Categories}

\begin{documentation}
 The corresponding \GAP property is given by
 \begin{center}
  \texttt{IsSymmetricClosedMonoidalCategory}.  
 \end{center}
\end{documentation}

\begin{definition}[\cite{MLCWM}]
 A symmetric monoidal category $\mathbf{C}$
 which has for each functor $- \otimes b: \mathbf{C} \rightarrow \mathbf{C}$
 a right adjoint (denoted by $\underline{\Hom}(b,-)$)
 is called a \textbf{symmetric closed monoidal category} or simply \textbf{closed category}.
\end{definition}

\begin{remark}
 The family of right adjoints defines a bifunctor
 $\underline{\Hom}(-,-): \mathbf{C}^{\mathrm{op}} \times \mathbf{C} \rightarrow \mathbf{C}$
 which we consider as a part of the datum of a closed category.
\end{remark}


Basic operations for symmetric closed monoidal categories coming from the existential quantifiers
in the definition
(in addition to the basic operations for symmetric monoidal categories):
\begin{itemize}
  \item \texttt{InternalHomOnObjects}
  \item \texttt{InternalHomOnMorphisms}
  \item \texttt{EvaluationMorphism}
  \item \texttt{CoevaluationMorphism}
\end{itemize}

\subsection{Rigid Symmetric Closed Monoidal Categories}

\begin{documentation}
 The corresponding \GAP property is given by
 \begin{center}
  \texttt{IsRigidSymmetricClosedMonoidalCategory}.  
 \end{center}
\end{documentation}

\subsection{Strict Monoidal Categories}

\begin{documentation}
 The corresponding \GAP property is given by
 \begin{center}
  \texttt{IsStrictMonoidalCategory}.  
 \end{center}
\end{documentation}

\section{Derivations}\label{section:derivations}