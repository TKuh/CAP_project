% \usepackage[latin1]{inputenc} 
\usepackage[utf8]{inputenc}
\usepackage[T1]{fontenc}
\usepackage[english]{babel}
\usepackage{a4wide}

\usepackage{mathabx}
\usepackage{mathrsfs}

\usepackage{latexsym}
\usepackage{amssymb}
\usepackage{amsthm}

\usepackage{colortbl}

\usepackage[all]{xy}
\usepackage{verbatim}
\usepackage{listings}
\usepackage{fancyvrb}

\usepackage{graphicx}

\usepackage[dvipsnames]{xcolor}
\usepackage{accents} %% \undertilde

\usepackage{enumerate}
\usepackage{wrapfig}

\usepackage{tikz}
\usepackage{tikz-cd}
\usetikzlibrary{shapes,arrows,matrix,backgrounds,positioning,plotmarks,calc,patterns,matrix,decorations.pathreplacing,decorations.pathmorphing,decorations.text}
\usepackage[colorinlistoftodos,shadow]{todonotes}

\usepackage{multirow}
\usepackage{mdwlist}

\usepackage{stmaryrd}
\usepackage{mathdots} %for iddots

\usepackage{fancyvrb}

\usepackage[toc,page]{appendix}
\usepackage{float}

\usepackage{extarrows}
\usepackage{enumitem}

\newtheoremstyle{mytheoremstyle} % name
    {5pt}                    % Space above
    {5pt}                    % Space below
    {\itshape}                   % Body font
    {\parindent}                           % Indent amount (empty = no indent, \parindent = para indent)
    {\bf}                   % Theorem head font
    {.}                          % Punctuation after theorem head
    {.5em}                       % Space after thm head: " " = normal interword space; \newline = linebreak
    {}  % Theorem head spec (can be left empty, meaning ‘normal’)
    
\newtheoremstyle{mytdefintionstyle} % name
    {5pt}                    % Space above
    {5pt}                    % Space below
    {\rm}                   % Body font
    {\parindent}                           % Indent amount
    {\bf}                   % Theorem head font
    {.}                          % Punctuation after theorem head
    {.5em}                       % Space after theorem head
    {}  % Theorem head spec (can be left empty, meaning ‘normal’)

\theoremstyle{mytheoremstyle}

\newtheorem{theorem}{Theorem}[section]
\newtheorem{conjecture}[theorem]{Conjecture}
\newtheorem{lemma}[theorem]{Lemma}
\newtheorem{proposition}[theorem]{Proposition}
\newtheorem{corollary}[theorem]{Corollary}
\newtheorem{algorithm}[theorem]{Algorithm}


\theoremstyle{mytdefintionstyle}

\newtheorem{axiom}[theorem]{Axiom}
\newtheorem{definition}[theorem]{Definition}
\newtheorem{example}[theorem]{Example}
\newtheorem*{notation}{Notation}
\newtheorem{type}[theorem]{Typing}
\newtheorem{documentation}[theorem]{Documentation}
\newtheorem{computation}[theorem]{Computation}
\newtheorem*{convention}{Convention}
\newtheorem{sequent}[theorem]{Sequent}
\newtheorem*{specification}{Specification}

\theoremstyle{remark}
\newtheorem{remark}[theorem]{Remark}
\newtheorem{question}[theorem]{Question}


%% \theoremstyle{definition}


\newtheoremstyle{exmp_contd} 
{\topsep} {\topsep}% 
{\upshape}% Body font 
{}% Indent amount (empty = no indent, \parindent = para indent) 
{\bfseries}% Thm head font
{}% Punctuation after thm head 
{ }% Space after thm head (\newline = linebreak) 
{\thmname{#1}\,\thmnumber{ #2}\thmnote{#3}\enspace(continued)}% Thm head spec 

\theoremstyle{exmp_contd} 
\newtheorem*{exmp_contd}{Example}

%\makeindex
\usepackage{xspace}

\DeclareMathOperator{\Spec}{Spec}
\DeclareMathOperator{\GSpec}{\underline{Spec}}
\DeclareMathOperator{\Proj}{Proj}

\newcommand{\CapPkg}{\textsc{Cap}\xspace}
\newcommand{\GAP}{\texttt{GAP}\xspace}
\newcommand{\Length}{\mathrm{Length}}
\newcommand{\FailedObj}{\mathrm{FailedObj}}
\newcommand{\FailedMor}{\mathrm{FailedMor}}
\newcommand{\TerminalObject}{\mathrm{TerminalObject}}
\newcommand{\Eval}{\mathrm{Eval}}
\newcommand{\UniversalMorphismIntoTerminalObject}{\mathrm{UniversalMorphismIntoTerminalObject}}
\newcommand{\Source}{\mathrm{Source}}
\newcommand{\Range}{\mathrm{Range}}
\newcommand{\PreCompose}{\mathrm{PreCompose}}
\newcommand{\IsIdenticalObj}{\mathrm{IsIdenticalObj}}
\newcommand{\Prop}{\mathrm{Prop}}
\newcommand{\Type}{\mathrm{Type}}
\newcommand{\Bool}{\mathrm{Bool}}

\newcommand{\ListObj}{\mathrm{ListObj}}
\newcommand{\ListMor}{\mathrm{ListMor}}
\newcommand{\ListInt}{\mathrm{ListInt}}
\newcommand{\Int}{\mathrm{Int}}
\newcommand{\IsTerminal}{\mathrm{IsTerminal}}
\newcommand{\IsInitial}{\mathrm{IsInitial}}
\newcommand{\IsFree}{\mathrm{IsFree}}
\newcommand{\IsZero}{\mathrm{IsZero}}
\newcommand{\IsInjective}{\mathrm{IsInjective}}
\newcommand{\IsProjective}{\mathrm{IsProjective}}
\newcommand{\DirectProduct}{\mathrm{DirectProduct}}
\newcommand{\DirectSum}{\mathrm{DirectSum}}
\newcommand{\IsIsomorphism}{\mathrm{IsIsomorphism}}
\newcommand{\IdentityMorphism}{\mathrm{IdentityMorphism}}
\newcommand{\IsSplitMonomorphism}{\mathrm{IsSplitMonomorphism}}
\newcommand{\IsSplitEpimorphism}{\mathrm{IsSplitEpimorphism}}
\newcommand{\IsOne}{\mathrm{IsOne}}
\newcommand{\IsAutomorphism}{\mathrm{IsAutomorphism}}
\newcommand{\IsMonomorphism}{\mathrm{IsMonomorphism}}
\newcommand{\IsEpimorphism}{\mathrm{IsEpimorphism}}
\newcommand{\IsEndomorphism}{\mathrm{IsEndomorphism}}
\newcommand{\ProjectionInFactor}{\mathrm{ProjectionInFactor}}
\newcommand{\ProjectionInFactorOfDirectProduct}{\mathrm{ProjectionInFactorOfDirectProduct}}
\newcommand{\ProjectionInFactorOfFiberProduct}{\mathrm{ProjectionInFactorOfFiberProduct}}
\newcommand{\InjectionOfCofactorOfPushout}{\mathrm{InjectionOfCofactorOfPushout}}
\newcommand{\InjectionOfCofactorOfCoproduct}{\mathrm{InjectionOfCofactorOfCoproduct}}
\newcommand{\IsEqualForMorphisms}{\mathrm{IsEqualForMorphisms}}
\newcommand{\IsEqualForObjects}{\mathrm{IsEqualForObjects}}
\newcommand{\FiberProduct}{\mathrm{FiberProduct}}
\newcommand{\KernelObject}{\mathrm{KernelObject}}
\newcommand{\KernelEmb}{\mathrm{KernelEmb}}
\newcommand{\KernelLift}{\mathrm{KernelLift}}
\newcommand{\CokernelColift}{\mathrm{CokernelColift}}
\newcommand{\UniversalMorphismIntoFiberProduct}{\mathrm{UniversalMorphismIntoFiberProduct}}
\newcommand{\UniversalMorphismFromPushout}{\mathrm{UniversalMorphismFromPushout}}
\newcommand{\UniversalMorphismIntoDirectProduct}{\mathrm{UniversalMorphismIntoDirectProduct}}
\newcommand{\Pushout}{\mathrm{Pushout}}
\newcommand{\InjectionOfCofactor}{\mathrm{InjectionOfCofactor}}
\newcommand{\Equalizer}{\mathrm{Equalizer}}
\newcommand{\Coequalizer}{\mathrm{Coequalizer}}
\newcommand{\Cokernel}{\mathrm{Cokernel}}
\newcommand{\Inverse}{\mathrm{Inverse}}
\newcommand{\CokernelProj}{\mathrm{CokernelProj}}
\newcommand{\Coproduct}{\mathrm{Coproduct}}
\newcommand{\UniversalMorphismFromCoproduct}{\mathrm{UniversalMorphismFromCoproduct}}

\newcommand{\CC}{\mathbf{C}}
\newcommand{\A}{\mathbb{A}}
\newcommand{\N}{\mathbb{N}}
\newcommand{\Z}{\mathbb{Z}}
\newcommand{\Ham}{\mathbb{H}}
\newcommand{\Q}{\mathbb{Q}}
\newcommand{\R}{\mathbb{R}}
\newcommand{\C}{\mathbb{C}}
\newcommand{\F}{\mathbb{F}}
\newcommand{\V}{\mathbb{V}}
\newcommand{\Pro}{\mathbb{P}}
\newcommand{\G}{\mathbb{G}}
\newcommand{\Set}{\mathbf{Set}}
\newcommand{\Top}{\mathbf{Top}}
\newcommand{\Grp}{\mathbf{Grp}}
\newcommand{\Schemes}{\mathbf{Schemes}}


\newcommand{\CH}{\mathrm{H}}

\newcommand{\PSh}{\mathfrak{PSh}}
\newcommand{\Sh}{\mathfrak{Sh}}
\newcommand{\Coh}{\mathfrak{Coh}}
\newcommand{\qcCoh}{\mathfrak{qCoh}}
\newcommand{\CTate}{\mathfrak{Tate}}
\newcommand{\CohTate}{\mathfrak{CoreTate}}
\newcommand{\Lin}{\mathrm{Lin}}
\newcommand{\Ch}{\mathrm{Ch}}
\newcommand{\Der}{\mathrm{D}}
\newcommand{\FTate}{\mathrm{Tate}}

\newcommand{\mor}{\mathrm{mor}}
\newcommand{\Mor}{\mathrm{Mor}}
\newcommand{\obj}{\mathrm{obj}}
\newcommand{\Obj}{\mathrm{Obj}}
\newcommand{\Var}{\mathrm{Var}}

\newcommand{\im}{\mathrm{im}}

\newcommand{\maximum}{\mathrm{max}}
\newcommand{\minimum}{\mathrm{min}}

\newcommand{\colim}{\mathrm{colim}}


\newcommand{\lcm}{\mathrm{lcm}}
\newcommand{\trace}{\mathrm{trace}}

\newcommand{\reg}{\mathrm{reg}}

\newcommand{\Cone}{\mathrm{Cone}}
\newcommand{\Tang}{\mathrm{T}}
\newcommand{\LieG}{\mathfrak{g}}
\newcommand{\LieH}{\mathfrak{h}}
\newcommand{\LieT}{\mathfrak{t}}

\newcommand{\cha}{\mathrm{char}}
\DeclareMathOperator{\Irr}{\mathrm{Irr}}
\newcommand{\id}{\mathrm{id}}
\DeclareMathOperator{\Aut}{\mathrm{Aut}}
\DeclareMathOperator{\End}{\mathrm{End}}
\DeclareMathOperator{\Hom}{\mathrm{Hom}}
\DeclareMathOperator{\Iso}{\mathrm{Iso}}

\DeclareMathOperator{\IHom}{\mathit{Hom}}
\DeclareMathOperator{\gr}{\mathrm{gr}}


\DeclareMathOperator{\Bil}{\mathrm{Bil}}
\DeclareMathOperator{\kernel}{\mathrm{ker}}
\DeclareMathOperator{\cokernel}{\mathrm{coker}}
\DeclareMathOperator{\soc}{\mathrm{soc}}
\DeclareMathOperator{\rad}{\mathrm{rad}}
\DeclareMathOperator{\ad}{\mathrm{ad}}
\DeclareMathOperator{\Cent}{\mathrm{C}}
\DeclareMathOperator{\grDeriv}{\mathrm{grDer}}
\DeclareMathOperator{\Deriv}{\mathrm{Der}}

\DeclareMathOperator{\Div}{\mathrm{Div}}
\DeclareMathOperator{\CDiv}{\mathrm{CDiv}}
\DeclareMathOperator{\Cl}{\mathrm{Cl}}
\DeclareMathOperator{\CaCl}{\mathrm{CaCl}}
\DeclareMathOperator{\Pic}{\mathrm{Pic}}
\DeclareMathOperator{\Supp}{\mathrm{Supp}}
\DeclareMathOperator{\Ouv}{\mathbf{Ouv}}
\DeclareMathOperator{\Kzero}{\mathrm{K_{0}}}


\DeclareMathOperator{\trdeg}{\mathrm{trdeg}}

\newcommand{\dUniv}{d}

\DeclareMathOperator{\codim}{\mathrm{codim}}

\DeclareMathOperator{\HF}{\mathrm{HF}}
\DeclareMathOperator{\HPS}{\mathrm{H}}
\DeclareMathOperator{\HP}{\mathrm{HP}}

\DeclareMathOperator{\Sym}{\mathrm{Sym}}

\DeclareMathOperator{\Tor}{\mathrm{Tor}}
\DeclareMathOperator{\Ext}{\mathrm{Ext}}

% \DeclareMathOperator{\Source}{\mathrm{Source}}
% \DeclareMathOperator{\Range}{\mathrm{Range}}
% \DeclareMathOperator{\PreCompose}{\mathrm{PreCompose}}
\DeclareMathOperator{\Id}{\mathrm{Id}}

\DeclareMathOperator{\Sp}{\mathrm{Sp}}
\DeclareMathOperator{\SO}{\mathrm{SO}}
\DeclareMathOperator{\SL}{\mathrm{SL}}
\DeclareMathOperator{\slie}{\mathfrak{sl}}
\DeclareMathOperator{\PSL}{\mathrm{PSL}}
\DeclareMathOperator{\PGL}{\mathrm{PGL}}
\DeclareMathOperator{\GL}{\mathrm{GL}}
\DeclareMathOperator{\gl}{\mathfrak{gl}}
\newcommand{\GG}{\mathrm{G_{0}}}

\DeclareMathOperator{\Tot}{\mathrm{Tot}^{\oplus}}

\newcommand{\RF}{\mathbf{R}}
\newcommand{\LF}{\mathbf{L}}
\newcommand{\IdF}{\mathrm{Id}}



\DeclareMathOperator{\RepRing}{\mathrm{R}}
\DeclareMathOperator{\length}{\mathrm{length}}

\newcommand{\I}{\mathrm{I}}

\newcommand{\grmod}{\mathrm{grmod}}
\newcommand{\modC}{\mathrm{mod}}
\newcommand{\ModC}{\mathrm{Mod}}
\newcommand{\grMod}{\mathrm{grMod}}

\newcommand{\Borel}{\textsc{Borel}\xspace}
\newcommand{\Galois}{\textsc{Galois}\xspace}
\newcommand{\Gabriel}{\textsc{Gabriel}\xspace}
\newcommand{\Hilbert}{\textsc{Hilbert}\xspace}
\newcommand{\Poincare}{\textsc{Poincar\'e}\xspace}
\newcommand{\Switzer}{\textsc{Switzer}\xspace}
\newcommand{\Atiyah}{\textsc{Atiyah}\xspace}
\newcommand{\Rees}{\textsc{Rees}\xspace}
\newcommand{\Horrocks}{\textsc{Horrocks}\xspace}
\newcommand{\Mumford}{\textsc{Mumford}\xspace}
\newcommand{\Serre}{\textsc{Serre}\xspace}
\newcommand{\Hartshorne}{\textsc{Hartshorne}\xspace}
\newcommand{\Tate}{\textsc{Tate}\xspace}
\newcommand{\Bernstein}{\textsc{Bernstein}\xspace}
\newcommand{\Gelfand}{\textsc{Gel'fand}\xspace}
\newcommand{\Betti}{\textsc{Betti}\xspace}
\newcommand{\Castelnuovo}{\textsc{Castelnuovo}\xspace}
\newcommand{\Schwarzenberger}{\textsc{Schwarzenberger}\xspace}
\newcommand{\Vogelaar}{\textsc{Vogelaar}\xspace}
\newcommand{\Barth}{\textsc{Barth}\xspace}
\newcommand{\Elencwajg}{\textsc{Elencwajg}\xspace}
\newcommand{\Decker}{\textsc{Decker}\xspace}
\newcommand{\Schreyer}{\textsc{Schreyer}\xspace}
\newcommand{\Kodaira}{\textsc{Kodaira}\xspace}
\newcommand{\Artin}{\textsc{Artin}\xspace}
\newcommand{\Grothendieck}{\textsc{Grothendieck}\xspace}
\newcommand{\Yoneda}{\textsc{Yoneda}\xspace}
\newcommand{\Noether}{\textsc{Noether}\xspace}
\newcommand{\Nakayama}{\textsc{Nakayama}\xspace}
\newcommand{\Abel}{\textsc{Abel}\xspace}
\newcommand{\Koszul}{\textsc{Koszul}\xspace}
\newcommand{\Reynolds}{\textsc{Reynolds}\xspace}
\newcommand{\Kronecker}{\textsc{Kronecker}\xspace}
\newcommand{\Frobenius}{\textsc{Frobenius}\xspace}
\newcommand{\Heisenberg}{\textsc{Heisenberg}\xspace}
\newcommand{\Schur}{\textsc{Schur}\xspace}
\newcommand{\Taylor}{\textsc{Taylor}\xspace}
\newcommand{\Jacobson}{\textsc{Jacobson}\xspace}
\newcommand{\Jacobi}{\textsc{Jacobi}\xspace}
\newcommand{\Euler}{\textsc{Euler}\xspace}
\newcommand{\Euclid}{\textsc{Euclid}\xspace}

\newcommand{\Weil}{\textsc{Weil}\xspace}
\newcommand{\Cartier}{\textsc{Cartier}\xspace}
\newcommand{\Cartan}{\textsc{Cartan}\xspace}
\newcommand{\Engel}{\textsc{Engel}\xspace}
\newcommand{\Picard}{\textsc{Picard}\xspace}
\newcommand{\Veronese}{\textsc{Veronese}\xspace}
\newcommand{\Segre}{\textsc{Segre}\xspace}
\newcommand{\Pluecker}{\textsc{Pluecker}\xspace}
\newcommand{\Hodge}{\textsc{Hodge}\xspace}
\newcommand{\Bezout}{\textsc{B\'ezout}\xspace}
\newcommand{\Enriques}{\textsc{Enriques}\xspace}
\newcommand{\Morita}{\textsc{Morita}\xspace}
\newcommand{\Leibnitz}{\textsc{Leibnitz}\xspace}
\newcommand{\Lie}{\textsc{Lie}\xspace}
\newcommand{\Zariski}{\textsc{Zariski}\xspace}
\newcommand{\Killing}{\textsc{Killing}\xspace}
\newcommand{\Casimir}{\textsc{Casimir}\xspace}
\newcommand{\Jordan}{\textsc{Jordan}\xspace}
\newcommand{\Chevalley}{\textsc{Chevalley}\xspace}
\newcommand{\Campbell}{\textsc{Campbell}\xspace}
\newcommand{\Lindeloef}{\textsc{Lindel\"of}\xspace}
\newcommand{\Freyd}{\textsc{Freyd}\xspace}
\newcommand{\Mitchell}{\textsc{Mitchell}\xspace}

\newcommand{\Kummer}{\textsc{Kummer}\xspace}
\newcommand{\Hausdorff}{\textsc{Hausdorff}\xspace}

\newcommand{\Sfpres}{S-\mathrm{fpres}}

% Einige Huepfskripte; zum kurzzeitigen Verlassen der Matrix
% setzt Text relativ zur position #2=x, #3=y
\newcommand{\jumpxy}[3]{\save[]+<#2, #3>*{#1} \restore}
%#2 = rel. Richtung
\newcommand{\jumpdir}[2]{\save[]+#2*{#1} \restore}
%z.B. ist \jumpdir{TEXT}{/:a(35) +1cm/} moeglich usw....

\setlength{\marginparwidth}{2cm}
%\reversemarginpar

\newcommand{\itembold}[1]{\item {\bf #1:}}
