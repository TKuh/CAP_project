In this chapter, we explain the philosophy of constructive category theory 
underlying our software project \CapPkg.
In short: Existential quantifiers have to be turned into algorithms.
Using dependent type theory \cite{hottbook}, this short statement can be made precise.

\section{Propositions as Types}
Like ZFC, dependent type theory is a formal language in which we can do mathematics.
Every correct mathematical theorem stated in ``everyday language'' can in principle be
``compiled'' to a formal statement in dependent type theory, where its correctness
can then be verified formally.
One of the nice features of dependent type theory is called \textit{proposition as types}.
Unlike ZFC, dependent type theory makes no distinction between a proposition and
the objects which we want to describe (e.g. sets or types). So we shall think about
a proposition like
\[
 \exists x \in \Z: x^2 = 4
\]
as being the type (or for simplicity the set) consisting of all pairs $(x, p)$ where
$x$ is a solution of $x^2 = 4$, and $p$ is a witness
of this fact. This is why in dependent type theory, such a proposition
is written as a dependent sum (or for simplicity a disjoint union)
\[
\sum_{x: \Z} x^2 = 4.
\]
Proving a proposition $P$ in dependent type theory means exhibiting a term of $P$.
So for proving $\sum_{x: \Z} x^2 = 4$, we have to give an explicit $x:\Z$, e.g. $2$ or $-2$,
and have to show that $x^2 = 4$ (by unwrapping the definition of squaring an integer).

\section{Models on the Computer}

A proof of $\sum_{x: \Z} x^2 = 4$ with a proof assistant uses the constructions
that can be done within dependent type theory. This means that once a proof is found,
this proof is valid for every model of dependent type theory
(there are models interpreting types as sets or groupoids \cite{Hofmann96thegroupoid}).
Such models can be realized on the computer. For example, in the formalization 
of $\CapPkg$, we use \GAP-sets and \GAP-maps. A \GAP-map essentially is a
computer program taking only elements of its source as its arguments
and returns a uniquely determined element of its range, respecting a notion of 
equality of elements. This is exactly what an algorithm is supposed to do.

\section{Examples of Constructions in Category Theory}

Combining dependent type theory and computer models, we get to the following 
concept of constructive category theory:
\begin{enumerate}
 \item We use dependent type theory for our formulation of category theory.
 \item We use a computer model of dependent type theory to interpret category theory.
\end{enumerate}
A definition of categories within dependent type theory can be found in \cite{hottbook}.
We give some examples of the consequences of this concept of constructiveness.

\begin{example}[Identities]
 For every object $a$ in a category $\mathbf{C}$, there exists a morphism
 $\id_a$ such that for all objects $b$ and all morphisms $\alpha \in \Hom( a, b )$, $\beta \in \Hom( b, a )$,
 the equations $\id_a \circ \beta = \beta$ and $\alpha \circ \id_a = \alpha$ hold.
 Written in the syntax of dependent type theory, this statement looks as follows:
 \[
  \prod_{a:\Obj} \sum_{\id_a: \Hom(a,a)} \prod_{\substack{ b: \Obj \\ \alpha: \Hom(a,b) \\ \beta: \Hom(b,a)}}( \id_a \circ \beta = \beta ) \times ( \alpha \circ \id_b = \alpha ).
 \]
 A term of this type is equal to a
 dependent function $f$, mapping an object $a$ to a pair $(\id_a, p)$,
 where $\id_a$ is a morphism and $p$ is a proof that this morphism acts like a unit.
 The proof $p$ can be omitted in our computer model because it is a mere proposition, so we end up
 with a dependent function $a \mapsto \id_a$, which is given in our computer model
 by an explicit algorithm.
\end{example}

\begin{example}[Kernels]
 Universal objects such as kernels are not merely objects. They are a collection of data.
 Given a morphism $\phi \in \Hom(a,b)$, a kernel of $\phi$ consists of
 \begin{itemize}
  \item an object $k$,
  \item a morphism $\iota: k \rightarrow a$ such that $\phi \circ \iota = 0$,
  \item a universal property, namely: For every other object $t$ and morphism $\tau: t \rightarrow a$
        such that $\phi \circ \tau = 0$, there exists a unique morphism $l: t \rightarrow k$
        such that $\tau = \iota \circ l$.
 \end{itemize}
 Written in the syntax of dependent type theory, the type of kernels of $\phi$
 looks as follows:
 \begin{align*}
  \mathrm{KernelsOf}(\phi):\equiv
  \sum_{k:\Obj} \sum_{\iota: \Hom(k,a)} (\phi \circ \iota = 0) \times \left( \prod_{ \substack{ t: \Obj \\ \tau: \Hom( t, a ) \\ p: \phi \circ \tau = 0}} 
  \sum_{l: \Hom( t, k )} (\tau = \iota \circ l) \times \mathrm{Uniqueness}( l ) \right)
 \end{align*}
 A term of this type is equal to a tupel $(k, \iota, q, u )$ consisting of an object $k$,
 a morphism $\iota: k \rightarrow a$, a witness $q$ of $\phi \circ \iota = 0$, and a dependent function $u$.
 This dependent function takes as arguments an object $t$, a morphism $\tau: t \rightarrow a$, and a witness $p$ of $\phi \circ \tau = 0$.
 The output is a triple consisting of morphism $l: t \rightarrow k$, a witness of $\tau = \iota \circ l$, and a witness of the uniqueness of $l$.
 Again by omitting mere propositions in our computer model,
 $u$ is interpreted by an explicit algorithm $(t, \tau) \mapsto l$.
 \\
 We say that a category has kernels if for all objects $a,b$ and morphisms $\phi \in \Hom( a, b )$,
 $\phi$ has a kernel.
 Written in the syntax of dependent type theory, this statement looks as follows:
 \[
  \prod_{ \substack{ a,b: \Obj \\ \phi: \Hom(a,b) } } \mathrm{KernelsOf}( \phi ).
 \]
 A term of this type is equal to a dependent function mapping a triple $(a,b,\phi)$
 to a term of $\mathrm{KernelsOf}( \phi )$.
 Interpreted in our computer model, such a dependent function is an algorithm 
 $(a,b,\phi) \mapsto (k, \iota, q, u )$. Note that $u$ (the universal property) is an algorithm by itself,
 here constructed by another algorithm.
\end{example}

In the following examples we omit the descriptions in dependent type theory
and focus on their interpretations in a computer model.

\begin{example}[Functors]
 A functor $F$ between two categories $\mathbf{C}$ and $\mathbf{D}$ consists of
 \begin{itemize}
  \item a function $F_{0}: \Obj_{\mathbf{C}} \rightarrow \Obj_{\mathbf{D}}$, 
  \item for $a,b \in \Obj_{\mathbf{C}}$, a function $F_{a,b}: \Hom_{\mathbf{C}}(a,b) \rightarrow \Hom_{\mathbf{D}}( F_0(a), F_0(b) )$
 \end{itemize}
 such that
 \begin{itemize}
  \item $F_{a,a}( \id_a ) = \id_{F_{0}(a)}$,
  \item for $a,b,c \in \Obj_{\mathbf{C}}$, $\alpha: a \rightarrow b$, $\beta: b \rightarrow c$, we have
        $F_{a,c}( \beta \circ \alpha ) = F_{b,c}( \beta ) \circ F_{a,b}( \alpha )$.
 \end{itemize}
 Interpreting this in a computer model, the action of the functor $F$ is given by two algorithms:
 \begin{itemize}
  \item On objects: The argument is an object $a \in \mathbf{C}$. The output is an object $F_0(a) \in \mathbf{D}$.
  \item On morphisms: The arguments are objects $a,b$ and a morphism $\phi: a \rightarrow b$. 
        The output is a morphism $F_{a,b}(\phi): F_0(a) \rightarrow F_0(b)$.
 \end{itemize}

\end{example}

\begin{example}[Natural Transformations]
 A natural transformation $\nu$ between two functors $F,G: \mathbf{C} \rightarrow \mathbf{D}$ consists of
 \begin{itemize}
  \item a dependent function mapping an object $a \in \mathbf{C}$ to a morphism $\nu_{a} \in \Hom_{\mathbf{D}}(Fa, Ga)$
 \end{itemize}
 such that 
 \begin{itemize}
  \item for $b \in \mathbf{C}$, $\alpha: a \rightarrow b$, there is an equality  $G(\alpha) \circ \nu_{a} = \nu_{b} \circ F(\alpha)$.
 \end{itemize}
 Interpreting this in a computer model, a natural transformation is an algorithm 
 having an object $a$ as an argument. The output is a morphism $\nu_a: Fa \rightarrow Ga$.
\end{example}

\begin{example}[Tensor Hom Adjunction]
 One of the axioms of a closed monoidal category $\mathbf{C}$ (e.g. the category of modules over a commutative ring $R$)
 says that
 \begin{itemize}
  \item for each $b \in \mathbf{C}$, the functor $- \otimes b$ has a right adjoint $\underline{\Hom}(b,-)$.
 \end{itemize}
 To understand this axiom constructively, we have to unwrap it.
 So $\underline{\Hom}(b,-)$ being a right adjoint to $- \otimes b$ means that
 \begin{itemize}
  \item $\exists$ a natural transformation $\mathrm{coev}^b_a: a \rightarrow \underline{\Hom}(b, a \otimes b)$, called \textbf{coevaluation},
  \item $\exists$ a natural transformation $\mathrm{ev}^b_a: \underline{\Hom}(b,a) \otimes b \rightarrow a$, called \textbf{evaluation},
 \end{itemize}
 such that the so called zig-zag relations\footnote{The zig-zag relations are $\mathrm{ev}^b_{a \otimes b} \circ (\mathrm{coev}^b_a \otimes b) = \id_{a \otimes b}$ and
  $\underline{\Hom}(b,\mathrm{ev}^b_a) \circ \mathrm{coev}^b_{\underline{\Hom}(b,a)} = \id_{\underline{\Hom}(b,a)}$.} hold.
 Interpreting this axiom in a computer model,
 it says that we have two algorithms:
 \begin{enumerate}
  \item The arguments are objects $b,a$. The output is a morphism $a \rightarrow \underline{\Hom}(b, a \otimes b)$.
  \item The arguments are objects $b,a$. The output is a morphism $\underline{\Hom}(b,a) \otimes b \rightarrow a$.
 \end{enumerate}
 Having these morphisms interpreted in our computer model, we are able to construct all the natural
 morphisms on the computer that can be constructed formally on the level of dependent type theory.
 For example, let $a$ be an object, and denote by $a^{\vee} = \underline{\Hom}(a,1)$ its dual.
 We want to construct the natural map $a \rightarrow (a^{\vee})^{\vee}$. For this, we
 \begin{enumerate}
  \item construct $\mathrm{coev}^{(a^{\vee})}_a: a \rightarrow \underline{\Hom}(a^{\vee}, a \otimes a^{\vee})$,
  \item construct $\underline{\Hom}(a^{\vee}, B_{a, a^{\vee}}): \underline{\Hom}(a^{\vee}, a \otimes a^{\vee}) \rightarrow \underline{\Hom}(a^{\vee}, a^{\vee} \otimes a)$,
  \item construct $\mathrm{ev}^{a}_1: a^{\vee} \otimes a \rightarrow 1$,
  \item construct $\underline{\Hom}( a^{\vee}, \mathrm{ev}^{a}_1 ): \underline{\Hom}(a^{\vee}, a^{\vee} \otimes a) \rightarrow (a^{\vee})^{\vee}$.
 \end{enumerate}
 The desired morphism is given by the composition of $(1), (2)$, and $(4)$. Note that
 this is an example of a derivation implemented in \CapPkg.
 In general, using the evaluation and the coevaluation, we can explicitly construct the bijection
 \[
  \Hom( a \otimes b, c ) \cong \Hom( a, \underline{\Hom}(b,c) ),
 \]
 and use this in turn for constructing natural morphisms.
\end{example}