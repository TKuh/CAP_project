In this chapter, we will have a closer look at the caching
which is used in \CapPkg. Almost every function in \CapPkg is cached,
this means, consecutive calls with the same arguments lead to identical results.

\section{The role of caches}

Caching in \CapPkg is mainly done for two reasons: speed and consistency. On the speed side, it is often
a good idea to store already computed results. Most computers have a lot of memory right now, so storing is
cheap. It also simpifies programming, since commands can be typed multiple times without worrying about potential
recomputing.

The consistency part is a bit more sophisticated. Categorical operations in \CapPkg are mathematical functions
with respect to some specific equality notion. This means the user has to make sure that the output
of equal input is equal. Caching can help if the function implemented for equality is not good enough
to ensure this. If equal input is given to a Categorical operation, the caches look up if there already
is a result for this input and returns it instead of a newly computed result.

\section{Types of caching}

In \CapPkg, we generally use two types of caching: The \GAP internal attribute/property caching and
the caches implemented in \CapPkg. They differ mainly in two aspects, namely the way arguments of
functions are compared and how many arguments a cached function can have. We want to compare the two
ways of caching.

\subsection{Internal attribute caching}
\begin{enumerate}
 \item Only for one argument functions
 \item Compares arguments by \texttt{IsIdenticalObj}
 \item Always stores the result
\end{enumerate}

\subsection{\CapPkg caching}
\begin{enumerate}
 \item Arbitrary argument function
 \item Compares arguments by \texttt{IsEqualForCache}, which can be implemented separately for different types of objects
 \item Stores the result, or just keeps a weak pointer, or can be disabled completely
\end{enumerate}

Those properties of the \CapPkg caches are needed. Of Course, \CapPkg makes use of the \GAP caches as well. If some categorical
operation is a \GAP attribute, it first looks at the \GAP attribute, then in its \CapPkg cache.

\section{Switchability of caches}

In \CapPkg, the caches have three different stages. They can be hard caches, weak caches, and turned off.
Hard caches mean that the result of a computation is stored forever (even though the input is not necessarily), weak means it is stored as long as
there is a pointer to the object, and turned off means there is no storing at all. These modes can be switched for
every category and every categorical operation at every time, but we suggest to set it at the creation of a category.
This setting can be manipulated by the methods \texttt{SetCachingToWeak}, \texttt{SetCachingToCrisp}, and \texttt{DeactivateCaching}.
All those methods take the category as first argument and the name of the categorical operation (e.g. \texttt{DirectSum}) as second.
Also, there are methods which set the ceaches for all operations of the category. Those only take the category as argument.
Their names are
\begin{itemize}
	\item \texttt{SetCachingOfCategoryWeak},
	\item \texttt{SetCachingOfCategoryCrisp}, and
	\item\texttt{DeactivateCachingOfCategory}.
\end{itemize}
