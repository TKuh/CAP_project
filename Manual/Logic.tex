\CapPkg provides two types of logic applied to objects and morphisms of a category.
In this chapter, the two layers are described together with some features of the logic.
We start by giving some general remarks, then describing the logic and the syntax explicitly.
After that, we give a description how to create your own logic files to the category.

\section{General remarks to logic in \CapPkg}

There are currently two types of logical propagation between \GAP objects implemented.
The first is a relation between different filters.

\begin{example}
 Every isomorphism is a monomorphism. So there is an implication between the properties
 \texttt{IsIsomorphism} and \texttt{IsMonomorphism}. Every time the property \texttt{IsIsomorphism}
 is set to true, the property \texttt{IsMonomorphism} is also true.
\end{example}

In \CapPkg, such a propagation is handled by \texttt{TrueMethods}. That means those relations
are static and cannot been turn off, but also \GAP knows about them. We will call this type of
implication the \textbf{first type}.

The second propagation is the propagation of predicates between input and output objects
of a certain method.

\begin{example}
  The composition of two monomorphisms is again a monomorphism. So, if for both input morphisms
  of the method \texttt{PreCompose} the property \texttt{IsMonomorphism} becomes set to \texttt{true},
  in the resulting morphism of \texttt{PreCompose}, \texttt{IsMonomorphism} is also set to \texttt{true}.
\end{example}

Internally, such propagations are carried out via \texttt{ToDoLists}. This means that the properties
do not need to be known when the mentioned operation is executed, but can be propagated later. As a drawback,
\GAP itself does not know about the relation before it is propagated, so it cannot take use of it. We will
call this type of implication the \textbf{second type}.

\subsection{Switchability of Logic}

Since the first type of relations are hard wired via \texttt{TrueMethods}, there are not
meant to be switched on or off depending on the category. Contrary to this, the logic defined
by the relations from operations can be turned off or on for a specific \CapPkg category.
This can be achieved by using the commands \texttt{CapCategorySwitchLogicOn} and \texttt{CapCategorySwitchLogicOff}.
Generally it is not a good idea to use these switches mid-computation, since the \texttt{ToDoListEntries} for the
already computed objects remain and might still be executed. If one wants no logic for a specific category,
it is always best to switch off logic at creation time of the category.

\subsection{Adding own logic files}

For a specific category, additional logic files can be added via the \CapPkg logic API. Those files are
plain TeX files, so it is easy to include them into some manual. For the syntax please see \ref{Section:LogicOne} and \ref{Section:LogicTwo}.
Once a category is created, a logic file can be added to the category.
For the first type of implications, this is done via the command
\texttt{AddTheoremFileToCategory}. The command takes the category as first argument, and the path to the file as
second. For the second type of implications, this is done via the command \texttt{AddPredicateImplicationFileToCategory}.
Again, this command takes the category as first argument and the path to the file as second.

\section{Logic by theorems} \label{Section:LogicOne}

A logical implication by theorem is simply an implication between \GAP filters. In the logic file, the implication looks like this:
\begin{Verbatim}[frame=single]
\begin{sequent}
\begin{align*}
  A:\Obj ~|~ \IsZero( A ) \vdash \IsTerminal( A )
\end{align*}
\end{sequent}
\end{Verbatim}
To clarify the syntax here, please note the following. Each logical part needs to be in a sequent environment, i.e., begin\{sequent\}
is important. Next, it needs to be in an align\* environment, as seen above. The first part is the declaration of the type of the \GAP object,
possible choices here are \texttt{\textbackslash Obj} or \texttt{\textbackslash Mor}. After that, a \texttt{|}, separating the declaration of the objects from the
theorem. After that, the implying filter should be separated by \texttt{\textbackslash vdash} from the implied filter. Note that the filters
need to be installed as commands or math operators, since it must match the name of the \GAP filter after the backslash.

\section{Logic by Predicate Implication} \label{Section:LogicTwo}

Syntax for predicate implications along methods is a bit more complex, since \CapPkg has categorical operations
having lists and integers as arguments. Those are reflected in the syntax of of the theorems which can be entered. We are
not going to discuss the whole syntax here, but refer to the files in the \texttt{LogicForCAP} directory in \CapPkg to have
a closer look at the possibilities. Instead, we just look at a few examples.

\begin{example}
We start with a simple example.
\begin{Verbatim}[frame=single]
\begin{sequent}
\begin{align*}
  \alpha:\Mor ~&|~ () \\
  &\vdash \IsMonomorphism\big( \KernelEmbedding( \alpha ) \big)
\end{align*}
\end{sequent}
\end{Verbatim}

This is one of the easiest declaration possible. There is only one variable defined, namely \texttt{\textbackslash alpha}, and without any conditions,
the result of the method \texttt{KernelEmbedding} is a monomorphism. No conditions are always marked by \texttt{()}. 
\end{example}

\begin{example}
We continue with something more serious.
\begin{small}
\begin{Verbatim}[frame=single]
\begin{sequent}
\begin{align*}
  L: \ListObj ~|~ \big(\forall x \in L: \IsTerminal(x)\big)
     \vdash \IsTerminal\big( \DirectProduct( L ) \big)
\end{align*}
\end{sequent}
\end{Verbatim}
\end{small}
Here we see a more advanced type of implication. There is only a list of objects defined, and the precondition needs to hold
for all objects in the list. This can be achieved using the syntax above.
\end{example}

Some general remarks on this propositions at the end: It is generally not possible nor necessary to use more then one
categorical operation in a theorem. Also, there is also the possibility to define an integer variable at the beginning, and
so accessing a particular entry in an input or output list.



