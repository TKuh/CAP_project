In \CapPkg, we want to model categories.
In order to do this correctly, we have to implement the basic operations
of a category with respect to specifications dictated by \CapPkg.
In this chapter, we will see in detail what these specifications are.

This chapter is organized as follows:

In the first section, we define the notion of a category, so that it becomes
clear what kind of mathematical object we want to model.

In the second section, we define \GAP sets and \GAP maps, in order to have semantics
for the basic operation symbols.

Afterwards, we begin with the enumeration of all specifications, which are
the equality specifications, typing specifications, and mathematical specifications.

We conclude with a statement about the importance of specifications.

\section{Categories}\label{section:categories}

Classically, a category consists of a class of objects, a set of morphisms, identity morphisms, and a composition function
satisfying some simple axioms. In $\CapPkg$, we use a slightly different notion of a category.

\begin{definition}\label{definition:CapCategory}
 A \CapPkg category $\mathbf{C}$ consists of the following data:
 \begin{enumerate}
  \item A set $\Obj_{\mathbf{C}}$ of \textbf{objects}.
  \item For every pair $a,b \in \Obj_{\mathbf{C}}$, a set $\Hom_{\mathbf{C}}( a, b )$ of \textbf{morphisms}.
  \item For every pair $a,b \in \Obj_{\mathbf{C}}$, an equivalence relation $\sim_{a,b}$ on $\Hom_{\mathbf{C}}( a, b )$
  called \textbf{congruence for morphisms}.
  \item For every $a \in \Obj_{\mathbf{C}}$, an \textbf{identity morphism} $\id_a \in \Hom_{\mathbf{C}}( a, a )$.
  \item For every triple $a, b, c \in \Obj_{\mathbf{C}}$, a \textbf{composition function}
  \[
   \circ: \Hom_{\mathbf{C}}( b, c ) \times \Hom_{\mathbf{C}}( a, b ) \rightarrow \Hom_{\mathbf{C}}( a, c )
  \]
  compatible with the congruence, i.e., 
  if $\alpha, \alpha' \in \Hom_{C}( a, b )$, 
  $\beta, \beta' \in \Hom_{C}( b, c )$,
  $\alpha \sim_{a,b} \alpha'$ 
  and $\beta \sim_{b,c} \beta'$, 
  then $\beta \circ \alpha \sim_{a,c} \beta' \circ \alpha'$.
  \item For all $a, b \in \Obj_{\mathbf{C}}$, 
        $\alpha \in \Hom_{\mathbf{C}}( a, b )$, 
        we have 
        \[
        \left( \id_{b} \circ \alpha \right) \sim_{a,b} \alpha
        \]
        and
        \[
        \alpha \sim_{a,b} \left( \alpha \circ \id_{a} \right).
        \]
  \item For all $a,b,c,d \in \Obj_{\mathbf{C}}$, 
        $\alpha \in \Hom_{\mathbf{C}}( a, b )$, 
        $\beta \in \Hom_{\mathbf{C}}( b, c )$, 
        $\gamma \in \Hom_{\mathbf{C}}( c, d )$,
        we have
        \[
        \left(( \gamma \circ \beta ) \circ \alpha \right) \sim_{a,d} \left( \gamma \circ ( \beta \circ \alpha ) \right)
        \]
 \end{enumerate}
\end{definition}

So the main differences between a $\CapPkg$ category and a classical category are:
\begin{enumerate}
 \item A $\CapPkg$ category has a set of objects, not a class.
 \item A $\CapPkg$ category has as an additional part of its datum a congruence for morphisms, and the
 axioms are stated with respect to this congruence, and not with respect to equality.
\end{enumerate}

We will see that the congruence for morphisms actually makes the implementation of some categories easier (for example
the category of presentations).

\begin{remark}
 Passing to the quotient sets $\Hom_{C}( a, b )/\sim_{a,b}$ gives rise to a classical category $\mathbf{D}$, because
 all constructions and axioms respect the congruence.
 It is usually the case that we actually want to study $\mathbf{D}$, but that it is easier to implement a $\CapPkg$ category $\mathbf{C}$
 giving rise to $\mathbf{D}$.
\end{remark}

\begin{remark}
 In terms of higher category theory, a $\CapPkg$ category is a $2$-category such that the $2$-morphism sets are either empty or a singleton.
\end{remark}

\begin{convention}
 Throughout this manual we will use \textit{category} as a short term for a \CapPkg category. 
 If we want to refer to the classical notion of a category (i.e. the one used in \cite{MLCWM}) we will use the term \textit{classical category}.
\end{convention}

\section{\GAP Sets and \GAP Maps}\label{section:gap_sets}

In this section, we will first define \GAP sets and \GAP maps and
then associate to such objects actual sets and maps. 
Such a translation between objects on the computer and mathematical objects
is necessary for modeling mathematics on the computer.

\begin{remark}
 For our definitions to be consistent, we have to fix
 a set $\mathcal{U}$ containing all \GAP objects up to \texttt{IsIdenticalObj}
 which are relevant for the discussion (e.g. within a current session). 
 We call such a $\mathcal{U}$ a \textbf{\GAP universe}.
 We further define the set $\Bool := \{ \texttt{true}, \texttt{false} \}$.
\end{remark}


\begin{definition}
 A \textbf{\GAP set} is a pair $(P, =_P)$ consisting of
 \begin{itemize}
  \item a \GAP function $P$
  \item a binary \GAP operation $=_P$
 \end{itemize}
both having values in $\Bool$, such that
\begin{enumerate}
 \item $P$ defines a map $\mathcal{U} \rightarrow \Bool$,
 \item $=_P$ defines a map $\{ ( a,b ) \in \mathcal{U} \times \mathcal{U} ~|~ P(a) = P(b) = \texttt{true} \} \rightarrow \Bool$,
 \item $=_P$ defines an equivalence relation $\sim_{=_P}$ on $\{ a \in \mathcal{U} ~|~ P(a) = \texttt{true} \}$.
\end{enumerate}
\end{definition}

\begin{notation}
 For a \GAP object $x$ such that $P(x) = \texttt{true}$, we simply write $x \in P$.
\end{notation}

\begin{definition}
 A \textbf{\GAP map} between \GAP sets $(P, =_P)$, $(Q, =_Q)$ is a \GAP operation $\Omega$, such that
 \begin{enumerate}
  \item for $p \in P$, we have $\Omega( p ) \in Q$,
  \item for $p =_P p'$, we have $\Omega( p ) =_Q \Omega( p')$,
  \item for $y \not\in P$, $\Omega( y )$ throws an error.
 \end{enumerate}
 For a natural number $n \in \N$, we define $n$-ary \GAP maps analogously, i.e., 
 they respect equality component-wise.
\end{definition}

\begin{remark}\label{remark:implementation_of_gap_maps}[Implementation of a \GAP map]
 Let $\Omega$ be a \GAP operation installed via the \GAP methods $\mu_1, \dots, \mu_n$.
 Then $\Omega$ defines a \GAP map $(P,=_P) \rightarrow (Q, =_Q)$ if and only if the following three conditions are satisfied:
 \begin{enumerate}
  \item $\mu_k( p ) =_Q \mu_l( p )$ for $p \in P$ and applicable methods $\mu_k, \mu_l$, where $k,l \in \{ 1 \dots n \}$.
  \item For every $p \in P$ there exists an $i \in \{ 1 \dots n \}$ such that $p$ lies in the filter of $\mu_i$ (i.e., 
  the filters of the methods cover $P$).
  \item $\Omega(y)$ throws an error for $y \not \in P$.
 \end{enumerate}
 The first item can always be technically realized by caching, but note that this is in general not the best strategy.
 The second item is satisfied in particular if there is a ``fallback method'' which is applicable for all $p \in P$.
 The third item will usually be automatically realized by $\CapPkg$ for all basic operations.
\end{remark}

In order to model concatenations of maps, we have to define a notion of concatenation of \GAP maps.

\begin{definition}
 A sequence of \GAP maps of length greater then one
 \[
 \Omega_1: (P_1, =_{\mathrm{P_1}}) \rightarrow (P_2, =_{\mathrm{P_2}}), ~\Omega_2: (P_2, =_{\mathrm{P_2}}) \rightarrow (P_3, =_{\mathrm{P_3}}), \dots,
 \]
 is called a \textbf{concatenation of \GAP maps}.
\end{definition}

\begin{example}
 If we would rely on weak caching in order to satisfy the first item in remark \ref{remark:implementation_of_gap_maps},
 a concatenation of \GAP maps would not necessarily yield equal output for equal input.
 Here is an example of what can go wrong:
 
 Let $R$ be a ring. We take matrices $M \in R^{n \times m}$ for integers $n,m$ as a data structure of
 modules over $R$, where the matrix $M$ shall represent the quotient module $R^{ 1 \times m }/ M$.
 Two modules are considered equal (\texttt{IsEqualForObjects}) if their representing matrices are equal.
 
 So for example, if we take $A = ( 0 )$ and $B = ( 1 )$, then $A$ represents a module isomorphic to $R$, 
 $B$ represents a module isomorphic to the zero module.
 A canonical way to implement the direct sum of two modules would be to construct a diagonal block matrix,
 so that applying $A \oplus -$ twice to $B$ yields
 \[
  \mathtt{result_1 := }
  A \oplus( A \oplus B ) =
  A \oplus
  \begin{pmatrix}
   0 & 0 \\
   0 & 1
  \end{pmatrix} =
  \begin{pmatrix}
   0 & 0 & 0 \\
   0 & 0 & 0 \\
   0 & 0 & 1
  \end{pmatrix}.
 \]
 If $A \oplus - $ works with a weak cache, the result of $A \oplus B$ might get lost, even though we still can access $\mathtt{result_1}$.
 Now assume that the attribute $\texttt{IsZero}$ becomes known for $B$, and assume furthermore that there is a special implementation
 for $A \oplus -$ which just returns $A$ in that case, we suddenly have
 \[
  \mathtt{result_2 := }
  A \oplus( A \oplus B ) =
  A \oplus A =
  \begin{pmatrix}
   0 & 0 \\
   0 & 0 
  \end{pmatrix},
 \]
leaving us with two saved results from syntactically identical expressions
having non equal evaluations.
\end{example}

We now associate to every \GAP set and every \GAP map a corresponding object in set theory.

\begin{definition}
 Let $(P, =_P)$ be a \GAP set. We define the \textbf{associated set} by
 \[
  \overline{P} := \{ x \in \mathcal{U} ~|~ x \in P \} / \sim_{ =_P }.
 \]
 The equivalence class of an object $x \in P$ is denoted by $\overline{x}$.
\end{definition}

\begin{definition}
 Let $\alpha: (P, =_P) \rightarrow (Q, =_Q)$ be a \GAP map. We define the \textbf{associated map} by
 \[
  \overline{\alpha} := \{ (\overline{p}, \overline{q}) ~|~ p \in P, q \in Q, \alpha( p ) =_{Q} q \} \subseteq \overline{P} \times \overline{Q}.
 \]
\end{definition}

The definitions of the associated $n$-ary maps for $n \in \N$ and for the associated concatenation map are similar.

\section{Equality Specifications}
\label{section:equality_specifications}

\begin{specification}
 Every basic operation has to yield equal output for given equal input.
 \end{specification}

Stated in the terminology of section \ref{section:gap_sets}: Every basic operation has to be a \GAP map.
\\ \\
Realizing a basic operation as a \GAP map means that we first have to understand its domain and codomain,
which are \GAP sets.
So let $\mathbf{C}$ be a category object, i.e., a category realized as a \GAP object in $\CapPkg$.
Every basic operation of $\mathbf{C}$ takes finitely many arguments as an input and computes one single output.
Each input argument has one of the following types:
\begin{enumerate}
 \item $\Int$, an integer.
 \item $\Bool$, a Boolean.
 \item $\Obj_{\mathbf{C}}$, an object of $\mathbf{C}$.
 \item $\Hom_{\mathbf{C}}( a, b )$, a morphism with source and range given by objects $a$ and $b$ in $\mathbf{C}$, respectively (this is a dependent type).
       We write $\Mor_{\mathbf{C}} := \sum_{a,b: \Obj_{\mathbf{C}}} \Hom_{\mathbf{C}}( a, b )$ for the type of all morphisms.
 \item $\ListObj_{\mathbf{C}}$, a finite list of objects.
 \item $\ListMor_{\mathbf{C}}$, a finite list of morphisms which possibly have different sources or ranges.
\end{enumerate}
These are also the possible output types except that an output never is a list. \\
We now specify what it means for two terms of the same type to be equal.
\begin{enumerate}
 \item For $\Int$ and $\Bool$, equality is given by the equality operations in $\GAP$.
 \item For $\ListObj_{\mathbf{C}}$ and $\ListMor_{\mathbf{C}}$, equality is given by entry-wise equality.
 \item For $\Obj_{\mathbf{C}}$, equality is given by the basic operation \texttt{IsEqualForObjects}.
 \item For $\Hom_{\mathbf{C}}( a, b )$, equality is given by the basic operation \texttt{IsEqualForMorphisms}.
\end{enumerate}

Like every basic operation, \texttt{IsEqualForObjects} and \texttt{IsEqualForMorphisms} can be added to 
a category object using the corresponding add functions \texttt{AddIsEqualForObjects} and \texttt{AddIsEqualForMorphisms}.

\begin{remark}
 The basic operations \texttt{IsEqualForObjects} and \texttt{IsEqualForMorphisms} play the role
 of the equality functions of the sets $\Obj_{\mathbf{C}}$ and $\Hom_{\mathbf{C}}(a,b)$ in definition
 \ref{definition:CapCategory}. In particular, \texttt{IsEqualForMorphisms} shall not be confused
 with congruence for morphisms.
\end{remark}

\begin{remark}\label{remark:realization_of_types}
 The above listed types are differently realized in \CapPkg.
 \begin{itemize}
  \item For $\Int$, we use the realization of \GAP.
  \item For $\Bool$, we use the realization of \GAP, but only the values \texttt{true} and \texttt{false}
  are allowed (exception: equality functions, see \ref{subsection:specifications_is_equal_for_objects} and \ref{subsection:specifications_is_equal_for_morphisms}).
  \item For $\Obj_{\mathbf{C}}$, there is a \GAP filter \texttt{ObjectFilter}($\mathbf{C}$).
  \item The type $\Mor$ is realized by \GAP objects within the filter \texttt{MorphismFilter}($\mathbf{C}$). 
  The dependent type $\Hom_{\mathbf{C}}( a, b )$ is realized by objects of type $\Mor$ having the objects $a$, $b$ set
 for their attributes \texttt{Source} and \texttt{Range}. 
  \item There are no special \GAP filters for $\ListObj_{\mathbf{C}}$ and $\ListMor_{\mathbf{C}}$, simply the \GAP filter \texttt{IsList}
  is used.
\end{itemize}
\end{remark}

\subsection{Specifications for \texttt{IsEqualForObjects}}\label{subsection:specifications_is_equal_for_objects}
The basic operation \texttt{IsEqualForObjects} has a special specification.

\begin{specification}[\texttt{IsEqualForObjects}]
~
 \begin{enumerate}
  \item Input: two objects ($\Obj$).
  \item Output: a Boolean ($\Bool$).
  \item \texttt{IsEqualForObjects} has to respect the \GAP function \texttt{IsIdenticalObj}, i.e., 
  for objects $a, a', b, b'$ such that \texttt{IsIdenticalObj}($a$, $a'$) and \texttt{IsIdenticalObj}($b$, $b'$) holds,
  \texttt{IsEqualForObjects}($a$, $b$) and \texttt{IsEqualForObjects}($a'$, $b'$) yield equal Booleans.
  \item \texttt{IsEqualForObjects} has to give rise to an equivalence relation on the set of all objects together
  with \texttt{IsEqualForObjects} as an equality function.
 \end{enumerate}
 In short: The pair $( \Obj, \texttt{IsEqualForObjects} )$ has to become a \GAP set, see section \ref{section:gap_sets}.
\end{specification}

\begin{remark}
 It is allowed for \texttt{IsEqualForObjects} to return \texttt{fail}. This is interpreted by \CapPkg as
 \textit{``I don't know if the two objects are equal''}.
\end{remark}

\subsection{Specifications for \texttt{IsEqualForMorphisms}}\label{subsection:specifications_is_equal_for_morphisms}
The basic operation \texttt{IsEqualForMorphisms} has a special specification.

\begin{specification}[$\mathtt{IsEqualForMorphisms}$]
Let $a, b: \Obj$. 
 \begin{enumerate}
  \item Input: two morphisms ($\Hom(a,b)$) (having equal sources and ranges).
  \item Output: a Boolean ($\Bool$).
  \item \texttt{IsEqualForMorphisms} has to respect the \GAP function \texttt{IsIdenticalObj}, i.e., 
  for morphisms $\alpha, \alpha', \beta, \beta'$ such that \texttt{IsIdenticalObj}($\alpha$, $\alpha'$) and \texttt{IsIdenticalObj}($\beta$, $\beta'$) holds,
  \texttt{IsEqualForMorphisms}($\alpha$, $\beta$) and \texttt{IsEqualForMorphisms}($\alpha'$, $\beta'$) yield equal Booleans.
  \item \texttt{IsEqualForMorphisms} has to give rise to an equivalence relation on the set of all morphisms with prescribed source and range together
  with \texttt{IsEqualForObjects} as an equality function.
 \end{enumerate}
 The pair $( \Hom(a,b), \texttt{IsEqualForMorphisms} )$ has to become a \GAP set
 (assuming we restrict \texttt{IsEqualForMorphisms} to $\Hom(a,b)$), see section \ref{section:gap_sets}.
\end{specification}

\begin{remark}
 It is allowed for \texttt{IsEqualForMorphisms} to return \texttt{fail}. This is interpreted by \CapPkg as
 \textit{``I don't know if the two morphisms are equal''}.
\end{remark}


\section{Typing Specifications}

\begin{specification}
 Every basic operation has to match its type.
\end{specification}

Every basic operation symbol has a type. To be more precise, it has a (dependent) function type.
We refer to the first chapter of \cite{hottbook} for a good summary of dependent type theory.

\begin{example}\label{example:simple_typing}
 The type of $\IsZeroForObjects$ is
 \[
  \Obj \rightarrow \Bool.
 \]
 Given a term of type $\Obj$, the basic operation symbol $\IsZeroForObjects$ returns a term of type $\Bool$.
\end{example}

\begin{example}\label{example:simple_typing_with_morphism_input}
 For $a,b: \Obj$, the type of $\KernelObject$ is
 \[
  \Hom(a,b) \rightarrow \Obj.
 \]
 Given a term of type $\Hom(a,b)$, the basic operation symbol $\KernelObject$ returns a term of type $\Obj$.
\end{example}

\begin{example}\label{example:dependent_typing}
 For $a,b: \Obj$, the type of $\KernelEmbedding$ is
 \[
  \prod_{\phi: \Hom(a,b)} \Hom( \KernelObject(\phi), a ).
 \]
 Given a term of type $\Hom(a,b)$, the basic operation symbol $\KernelEmbedding$ returns a term of type $\Hom( \KernelObject(\phi), a )$.
 Thus $\KernelEmbedding$ has a dependent function type, because the type of its output depends on the input term.
\end{example}

For a basic operation to match its typing specification, it has to
\begin{enumerate}
 \item accept only input specified by its type, and throw an error otherwise,
 \item compute only output specified by its type.
\end{enumerate}

%\todo{specify when not, for example for performance}
%There should be different modi in \CapPkg:
% 1) no type checking
% 2) medium type checking (default)
% 3) full type checking
The first item is (almost) always handled by \CapPkg.
If we add a basic operation to \CapPkg via an add function, \CapPkg ensures that every call of that
basic operation first triggers a type checking for the input.
\\
The meaning and implications of the second item have to be analyzed carefully.
\begin{example}[\ref{example:simple_typing} continued]
 The output has to be a Boolean (c.f. remark \ref{remark:realization_of_types}).
\end{example}

\begin{example}[\ref{example:simple_typing_with_morphism_input} continued]
 The output has to be an object (c.f. remark \ref{remark:realization_of_types}).
\end{example}

\begin{example}[\ref{example:dependent_typing}]
 The output type depends on the given input. Let $a$ denote the source, $b$ denote the range of the input morphism $\phi$.
 Then the range of the output morphism has to be equal (\texttt{IsEqualForObjects}) to $a$.
 The source of the output morphism has to be equal (\texttt{IsEqualForObjects}) to the output of $\KernelObject(\phi)$.
\end{example}

As seen in example \ref{example:dependent_typing}, basic operation symbols (like $\KernelObject$) can be part of
the type of other basic operations (like $\prod_{\phi: \Hom(a,b)} \Hom( \KernelObject(\phi), a )$).
To implement a coherent model of all basic operations, we have to take care that such dependencies are fulfilled.
Fortunately, \CapPkg provides tools to help us with this task.

\begin{example}
 Strategies of an implementation of $\KernelObject$ and $\KernelEmbedding$ with the correct types:
 \begin{enumerate}
  \item Only implement $\KernelEmbedding$. \CapPkg will automatically derive (c.f. section \ref{section:derivations}) the basic operation
        $\KernelObject$ coherently for you (as the source of $\KernelEmbedding$).
  \item Implement $\KernelObject$ and the helper basic operation $\KernelEmbeddingWithGivenKernelObject$.
        This is an operation getting as an input a morphism $\phi$ and the result of $\KernelObject( \phi )$,
        which we can then use to construct a morphism with $\KernelObject( \phi )$ as its source (c.f. \ref{in_text:with_given}).
        \CapPkg will automatically derive (c.f. section \ref{section:derivations}) the basic operation
        $\KernelEmbedding$ coherently for you (by calling $\KernelEmbeddingWithGivenKernelObject$ with the output of $\KernelObject$ as an input).
  \item Implement $\KernelObject$, $\KernelEmbedding$, and $\KernelEmbeddingWithGivenKernelObject$, and enable caching for these functions (c.f. chapter \ref{chapter:caching}).
        There is a redirection mechanism (see \ref{in_text:redirect}) of \CapPkg that guarantees a correct typing.
 \end{enumerate}
 Of course, you always can only implement $\KernelObject$ and $\KernelEmbedding$, and prove the correct typing manually.
\end{example}



\section{Mathematical Specifications}

\begin{specification}
 Every basic operation has to compute what it is mathematically supposed to compute.
\end{specification} 

In order to implement a coherent model of a category, the basic operations that we provide have to match all mathematical
specifications. We give an example to illustrate what this means.

\begin{example}
 We want to implement $\KernelObject$, $\KernelEmbedding$, and $\KernelLift$.
 Let us take a look at the definition of a kernel in a category $\mathbf{C}$. \\
 For a given morphism $\alpha: A \rightarrow B$ in $\mathbf{C}$, a kernel of $\alpha$ consists of three parts:
 \begin{enumerate}
  \item an object $K \in \mathbf{C}$,
  \item a morphism $\iota: K \rightarrow A$ such that $\alpha \circ \iota \sim_{K,B} 0$,
  \item a dependent function $u$ mapping each morphism $\tau: T \rightarrow A$ satisfying $\alpha \circ \tau \sim_{T,B} 0$ to
  a morphism $u(\tau): T \rightarrow K$ such that $\iota \circ u( \tau ) \sim_{T,A} \tau$. 
 \end{enumerate}
 The triple $( K, \iota, u )$ is called a \textbf{kernel of $\alpha$} if the morphisms $u( \tau )$ are uniquely determined up to
 congruence of morphisms. The situation can be depicted as follows:
 
 \begin{center}
\begin{tikzpicture}[label/.style={postaction={
      decorate,
      decoration={markings, mark=at position .5 with \node #1;}}}]
      \coordinate (r) at (1.5,0);
      \coordinate (u) at (0,0.75);
      
      \node (M) {$A$};
      \node (N) at ($(M)+(r)$) {$B$};
      \node (K) at ($(M)-(r)+(u)$) {$K$};
      \node (L) at ($(K)-2*(u)$) {$T$};
      \draw[->,thick] (M) -- node[above]{$\alpha$} (N);
      \draw[->, thick] (K) to [bend left] node[above] {$0$}(N);
      \draw[right hook->,thick] (K) -- node[above]{$\iota$} (M);
      \draw[->,thick] (L) -- node[above]{$\tau$} (M);
      \draw[->, thick] (L) to [bend right] node[below] {$0$}(N);
      \draw[->,dotted,thick] (L) -- node[left]{$u( \tau )$} (K);
\end{tikzpicture}
\end{center}
 
 We call $K, \iota, u( \tau ) $ a $\KernelObject$, $\KernelEmbedding$, $\KernelLift$, respectively.
 Thus for implementing these basic operations properly, they have to yield outputs which together form a kernel of $\phi$.
\end{example}

\begin{remark}
 We note that the universal property of the kernel is formulated with the notion of congruence for morphisms,
 not with equality of morphisms. This is again due to our notion of a category, see section \ref{section:categories}.
\end{remark}


\section{The Importance of Specifications}

Once we have implemented a category matching all specifications, we can benefit
from all the constructions $\CapPkg$ automatically performs for us. 
$\CapPkg$ is able to use categories in order to construct new ones. It is also
able to derive algorithms using the basic operations as building blocks.
The correctness of all these automatic constructions heavily depends on 
the correctness of its building blocks.
\begin{enumerate}
 \item Equality specifications: necessary to interpret basic operations in set theory.
 Not matching these specifications means dealing with randomness. 
 \item Typing specifications: necessary to safely concatenate basic operations.
 \item Mathematical specifications: necessary because we want to model our mathematical
 notions correctly.
\end{enumerate}


