
\begin{itemize}
 
 %% is there a better placeholder instead of *?
 \itembold{\GAP *} interpret * in the context of \GAP, e.g., a \GAP filter is
   a filter in the context of \GAP.
 
 \itembold{\GAP function} function object in \GAP.
 
 \itembold{\GAP operation} operation object in \GAP.
 
 \itembold{\GAP method} function installed for some operation via InstallMethod.
 
 \itembold{Add function} function with name \texttt{Add*}, which adds functions as methods
   to the operation *.
 
 \itembold{Basic operation} operation for which functions can be added to the category, e.g., \texttt{PreCompose}.
 
 \itembold{Categorical operation} basic operation which represents a categorical construction. \texttt{IsIdenticalToIdentity}
   is a basic operation which is not a Categorical operation.
 
 \itembold{Basic operation symbol} The variable name or string which represents a basic operation.
 
 \itembold{\texttt{WithGiven} operation} basic operation which has the string \texttt{WithGiven} in its basic operation symbol.
 
 \itembold{Primitive operation} basic operation in a category for which the functions are installed via Add functions, not via
   derivations.
 
 \itembold{Derived operation} basic operation in a category which is installed by the derivation mechanism of \CapPkg.
 
 \itembold{Category} \CapPkg category.
 
 \item{$\CapPkg$ \textbf{category:}} concept of category implemented in \CapPkg, see definition \ref{definition:CapCategory}.
 
 \itembold{Classical category} Category as described in \cite{MLCWM}.
 
 \itembold{Category object} \GAP objects which represents a \CapPkg category, not to be confused with the \GAP category.
 
 \itembold{Object} \GAP object which represents an object of a category.
 
 \itembold{Morphism} \GAP object which represents a Morphism of a category.
 
 \itembold{Twocell} \GAP object which represents a twocell of a 2-category.
 
 \item{\CapPkg\textbf{:}} always means the complete \CapPkg project.
 
\end{itemize}
