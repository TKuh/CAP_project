Although \CapPkg offers many features and many possibilities to tweak computations,
it is fairly simple to integrate already existing data types and algorithms in \GAP
into the \CapPkg setup. In this chapter we give you a hand at the very first steps, using some
examples and explaining the use and the effect of the commands in it.
If you are new to \GAP, you can look up the documentation of all following 
methods via \GAP's help system 
\footnote{See \href{https://www.gap-system.org/Manuals/doc/ref/chap2.html}{https://www.gap-system.org/Manuals/doc/ref/chap2.html}. 
It might be neccessary to create \CapPkg's documentation by running \texttt{make doc} in the home folder of \CapPkg.}.

\section{The category of groups}\label{section:groups}

When implementing a category in \CapPkg, one should always have the corresponding classical
category in mind. In this example we start with the category of groups. The objects in this
category will be the groups, the morphisms will be the homomorphisms of groups.
We start by creating a \CapPkg category with a suitable name:

\begin{Verbatim}[commandchars=!@\%,frame=single]
!color@blue%@gap>%!color@red%@ LoadPackage( "CAP" );%
true
!color@blue%@gap>%!color@red%@ grps := CreateCapCategory( "groups" );%
groups
\end{Verbatim}


We continue by telling the category how operations on the objects or morphisms are performed by adding
functions the category. First, there should be a compose method for morphisms. In \GAP
morphisms of groups are composed via the \texttt{*} operator, so we use it for \textrm{PreCompose}.

\begin{Verbatim}[commandchars=!@\%,frame=single]
!color@blue%@gap>%!color@red%@ AddPreCompose( grps, \* );%
\end{Verbatim}



Now the composition of two morphisms will be computed by the \texttt{\textbackslash *} operation.
Another thing every category needs is a function to compute the identity morphism.

\begin{Verbatim}[commandchars=!@\%,frame=single]
!color@blue%@gap>%!color@red%@ identity_func := grp -> GroupHomomorphismByImages( grp, grp );%
function( grp ) ... end
!color@blue%@gap>%!color@red%@ AddIdentityMorphism( grps, identity_func );%
\end{Verbatim}



The command used above exactly creates the identity morphism of a group.
Next, we tell \CapPkg that every output of a basic operation for the category \texttt{grps} should automatically be added to the category \texttt{grps}.
This step is necessary since in this example, we simply integrate already existing algorithms and data structures of \GAP to
\CapPkg's infrastructure.

\begin{Verbatim}[commandchars=!@\%,frame=single]
!color@blue%@gap>%!color@red%@ EnableAddForCategoricalOperations( grps );%
\end{Verbatim}


Since the category has now all the functions we wanted it to have, we can finalize it.

\begin{Verbatim}[commandchars=!@\%,frame=single]
!color@blue%@gap>%!color@red%@ Finalize( grps );%
true 
\end{Verbatim}



Finalizing a category is necessary after adding all the desired operations and before constructing objects
for it. We can now create a group and tell the system that the group should be an object in the category.
Note that \texttt{AddObject} can only be applied to \GAP objects lying in the filter
\texttt{IsAttributeStoringRep}.

\begin{Verbatim}[commandchars=!@\%,frame=single]
!color@blue%@gap>%!color@red%@ S3 := SymmetricGroup( 3 );%
Sym( [ 1 .. 3 ] ) 
!color@blue%@gap>%!color@red%@ AddObject( grps, S3 );%
!color@blue%@gap>%!color@red%@ S4 := SymmetricGroup( 4 );                                                                                                                                                                            %
Sym( [ 1 .. 4 ] ) 
!color@blue%@gap>%!color@red%@ AddObject( grps, S4 );%
\end{Verbatim}



Now, \CapPkg considers those groups as objects in the category \texttt{grps}, and we can
ask them about their category. Also, they are now part of the \GAP filter \texttt{IsCapCategoryObject}.

\begin{Verbatim}[commandchars=!@\%,frame=single]
!color@blue%@gap>%!color@red%@ CapCategory( S3 );%
groups
!color@blue%@gap>%!color@red%@ IsCapCategoryObject( S4 );%
true
\end{Verbatim}


Now manipulation of the groups is possible using the functions we have already provided to the category.
It is possible to construct the identity of a group, and compose it with itself.

\begin{Verbatim}[commandchars=!@\%,frame=single]
!color@blue%@gap>%!color@red%@ id_S3 := IdentityMorphism( S3 );%
IdentityMapping( Sym( [ 1 .. 3 ] ) )
!color@blue%@gap>%!color@red%@ PreCompose( id_S3, id_S3 );%
IdentityMapping( Sym( [ 1 .. 3 ] ) )
\end{Verbatim}


Of course, one can also create a morphism between \texttt{S3} and \texttt{S4} and add it to the category.
After that, we can also compose it with the identity morphism.

\begin{Verbatim}[commandchars=!@\%,frame=single]
!color@blue%@gap>%!color@red%@ S3_S4 := GroupHomomorphismByImages( S3, S4, GeneratorsOfGroup(S3) );%
[ (1,2,3), (1,2) ] -> [ (1,2,3), (1,2) ]
!color@blue%@gap>%!color@red%@ AddMorphism( grps, S3_S4 );%
!color@blue%@gap>%!color@red%@ PreCompose( id_S3, S3_S4 );%
[ (1,2,3), (1,2) ] -> [ (1,2,3), (1,2) ]
\end{Verbatim}


Please note that the constructors for objects and morphisms used in this example are the ones provided
by \GAP itself, and the only ``change'' done to the data structure was adding it to the category.
This is one of the design principles of \CapPkg. Already existing data structures and algorithms
can be integrated into the system with little to no effort, which makes it possible to integrate \CapPkg
into many existing projects.

This example was pretty basic and only a way to show the very basic structure of \CapPkg. The next example
will construct a more sophisticated category and some of the computational tools in \CapPkg.


\section{The skeletal category of rational vector spaces}

In this section, we implement and compute with the category of finite dimensional rational vector spaces
(i.e., finite dimensional vector spaces defined over $\Q$) as an
example of an abelian category. We are going
to realize this category as a skeletal category, i.e., the objects will be the non-negative integers $\N_0$,
representing the dimension of the vector space, and we use $a \times b$ matrices with entries in $\Q$ as morphisms
from $a \in \N_0$ to $b \in \N_0$ (row convention). We will denote this category by $\Svec$
(the ``S'' stands for \emph{skeletal}, the ``v'' is small since we deal with \emph{finite dimensional} vector spaces).

\subsection{Implementation}

The first part of this tutorial is taken from a file and not from
an interactive session. Please note that for the sake of completeness, all the necessary code to create the
functions is displayed. This might seem confusing or simply too much. If you only want to know how a \CapPkg category
is initialized, only look at the headers of the functions. Also, once the whole process of creation is finished,
there are examples for computations in the category. If you are primarily interested in the computational capabilities of \CapPkg, please
continue directly with Subsection \ref{subsection:skeletal_vec_computation}.

\begin{small}
\begin{Verbatim}[commandchars=!@\%,frame=single]
LoadPackage( "CAP" );

LoadPackage( "MatricesForHomalg" );

DeclareRepresentation( "IsHomalgRationalVectorSpaceRep",
                       IsCapCategoryObjectRep,
                       [ ] );

BindGlobal( "TheTypeOfHomalgRationalVectorSpaces",
        NewType( TheFamilyOfCapCategoryObjects,
                IsHomalgRationalVectorSpaceRep ) );

DeclareRepresentation( "IsHomalgRationalVectorSpaceMorphismRep",
                       IsCapCategoryMorphismRep,
                       [ ] );

BindGlobal( "TheTypeOfHomalgRationalVectorSpaceMorphism",
        NewType( TheFamilyOfCapCategoryMorphisms,
                IsHomalgRationalVectorSpaceMorphismRep ) );

DeclareAttribute( "Dimension",
                  IsHomalgRationalVectorSpaceRep );

DeclareOperation( "QVectorSpace",
                  [ IsInt ] );

DeclareOperation( "VectorSpaceMorphism",
                  [ IsHomalgRationalVectorSpaceRep, IsObject,
                    IsHomalgRationalVectorSpaceRep ] );

BindGlobal( "vecspaces", CreateCapCategory( "VectorSpaces" ) );

BindGlobal( "VECTORSPACES_FIELD", HomalgFieldOfRationals( ) );

\end{Verbatim}
\end{small}


We start by loading \texttt{CAP} and the package \texttt{RingsForHomalg} of the \texttt{homalg} project.
In contrast to the groups example of Section \ref{section:groups}, we won't use existing data structures for the objects and morphisms
in $\Svec$, but we will create them on our own. Thus, we declare
a \GAP category for $\Svec$, a \GAP category for the objects in $\Svec$ and a \GAP category for the morphisms in $\Svec$.

After that, we declare the dimension as an attribute for objects in $\Svec$,
and an underlying matrix as an attribute for morphisms in $\Svec$.

Then we declare constructors
for objects and morphisms. The object constructor only has one attribute, which is the dimension of the vector space.
The morphism constructor has three arguments. A vector space, which is the \texttt{Source} of the morphism, a matrix
representing the morphism, and a second vector space, which is the \texttt{Range} of the morphism. Note that every morphism
which should be added to the category needs the \GAP attributes \texttt{Source} and \texttt{Range}, which should be
objects of the same category.

After the declarations of the constructors we create a global category \GAP object, which will be the category $\Svec$.
The call of \texttt{CreateCapCategory} lets \CapPkg create a category with name
``\texttt{Skeletal category of rational vector spaces}'' which lies in the filter \texttt{IsSQVecCat}
such that every \GAP object which is an object of $\Svec$ lies in the filter \texttt{IsSQVecObj}
and every \GAP object which is a morphism of $\Svec$ lies in the filter \texttt{IsSQVecMor}.
We do not want to consider two cells in $\Svec$, so we simply pass the generic filter \texttt{IsCapCategoryTwoCell}.
Since we are going to create an Abelian category, we tell the system that we are doing this, to have all the computational
properties of an Abelian category.
Last, we bind the field $\Q$ to a global variable.

\begin{small}
\begin{Verbatim}[frame=single]
##
InstallMethod( QVectorSpace,
               [ IsInt ],
               
  function( dim )
    local space;
    
    space := rec( );
    
    ObjectifyWithAttributes( space, TheTypeOfHomalgRationalVectorSpaces,
                             Dimension, dim 
    );
    
    Add( vecspaces, space );
    
    return space;
    
end );

##
InstallMethod( VectorSpaceMorphism,
                  [ IsHomalgRationalVectorSpaceRep, IsObject,
                    IsHomalgRationalVectorSpaceRep ],
                  
  function( source, matrix, range )
    local morphism;

    if not IsHomalgMatrix( matrix ) then
    
      morphism := HomalgMatrix( matrix, Dimension( source ), Dimension( range ),
                                VECTORSPACES_FIELD );

    else

      morphism := matrix;

    fi;

    morphism := rec( morphism := morphism );
    
    ObjectifyWithAttributes( morphism,
                             TheTypeOfHomalgRationalVectorSpaceMorphism,
                             Source, source,
                             Range, range 
    );

    Add( vecspaces, morphism );
    
    return morphism;
    
end );
\end{Verbatim}
\end{small}

The first constructor creates the objects of $\Svec$.
Here, the \CapPkg command
\begin{itemize}
  \item \texttt{CreateCapCategoryObjectWithAttributes}
\end{itemize}
creates a \GAP object that resides as an object in the category
$\Svec$, with the attribute \texttt{Dimension} set to \texttt{dim}.

The second constructor creates the morphisms of $\Svec$.
Here, the \CapPkg command
\begin{itemize}
  \item \texttt{CreateCapCategoryMorphismWithAttributes}
\end{itemize}
creates a \GAP object that resides as a morphism in the category
$\Svec$ with attribute \texttt{UnderlyingMatrix} set to \texttt{underlying\_matrix},
and with source and range coming from the arguments of the constructor.

The resulting objects and morphisms of these constructors
are automatically part of the category $\Svec$, and after finalizing the category all the operations from the
category will be applicable to them.

We now continue by adding functions to the category, starting with the same operations as in the groups example (Section \ref{section:groups}).
If you want to test your implementation function by function,
please remember to use the command \texttt{Finalize( SQVec )}
before testing. 

\begin{small}
\begin{Verbatim}[frame=single]
AddIdentityMorphism( vecspaces,
                     
  function( obj )

    return VectorSpaceMorphism( obj,
          HomalgIdentityMatrix( Dimension( obj ), VECTORSPACES_FIELD ), obj );
    
end );

AddPreCompose( vecspaces,

  function( mor_left, mor_right )
    local composition;

    composition := mor_left!.morphism * mor_right!.morphism;

    return VectorSpaceMorphism( Source( mor_left ),
                  composition, Range( mor_right ) );

end );
\end{Verbatim}
\end{small}

These are straight forward implementations. Before we continue
adding additional operations, there is another thing we have to take care of. The category of vector spaces $\Svec$ we wanted
to create is a skeletal category, i.e., vector spaces of the same dimension are supposed to be equal. At the moment
two consecutive calls of the object constructor will not lead to equal objects, since there is no comparison function.
\CapPkg offers facilities to give correct comparison functions for objects and morphisms to the category. If no function
is given, the \GAP function \texttt{IsIdenticalObj} is used. However, giving the right equality function to the
category is important, please see Section \ref{section:equality_specifications} about equalities for details. We will now add equalities
for objects and morphisms. Objects will be compared by their dimension, for morphisms the matrices will be compared entry-wise.
The equalities in \CapPkg do not need to be \texttt{true} or \texttt{false}, but are also allowed to return \texttt{fail},
which will be interpreted as non decidable. This is used for example for categories of unbounded complexes.

\begin{small}
\begin{Verbatim}[commandchars=!@\%,frame=single]
AddIsEqualForObjects( vecspaces,
  
  function( vecspace_1, vecspace_2 )
    
    return Dimension( vecspace_1 ) = Dimension( vecspace_2 );
    
end );

AddIsEqualForMorphisms( vecspaces,

  function( a, b )
  
    return a!.morphism = b!.morphism;
  
end );
\end{Verbatim}
\end{small}

Next we implement some additional functions in the category of vector spaces. We start by defining functions for the kernel.
A proper implementation needs at least two functions. One for \texttt{KernelEmbedding}, which can compute the embedding of the kernel
into the source of the given morphism. From this function then automatically a function for \texttt{KernelObject} is derived, which
computes the embedding and returns the source of this morphism. The second one, \texttt{KernelLift}, implements the universal property of the kernel.
The diagram of the kernel looks like this:
\begin{center}
\begin{tikzpicture}[description/.style={fill=white,inner sep=2pt}]
  \matrix (m) [matrix of math nodes, row sep=3em,
               column sep=2.5em, text height=1.5ex, text depth=0.25ex]
  { \mathrm{KernelObject} \left( \alpha \right) & & \\
    & A & B \\
    T & & \\ };
  \path[->] (m-1-1) edge node[auto]{$\mathrm{KernelEmbedding} \left( \alpha \right)$} (m-2-2);
  \path[->] (m-2-2) edge node[auto]{$\alpha$} (m-2-3);
  \path[->] (m-3-1) edge node[auto]{$\tau$} (m-2-2);
  \path[->, dashed] (m-3-1) edge node[auto]{$\mathrm{KernelLift} \left( \alpha, T, \tau \right)$} (m-1-1);
  \path[->] (m-3-1) edge[bend right] node[auto]{$0$} (m-2-3);
\end{tikzpicture}
\end{center}

Please note that it is important that the range of the output morphism of the function we give to \texttt{KernelLift},
under equal first argument, must coincide with the source of the output morphism of the function given to \texttt{KernelEmbedding}.
In our case this simply means they need to have the same dimension.

\begin{small}
\begin{Verbatim}[frame=single]
AddKernelEmb( vecspaces,

  function( morphism )
    local kernel_emb, kernel_obj;
    
    kernel_emb := SyzygiesOfRows( morphism!.morphism );
    
    kernel_obj := QVectorSpace( NrRows( kernel_emb ) );
    
    return VectorSpaceMorphism( kernel_obj, kernel_emb, Source( morphism ) );
    
end );

AddKernelLift( vecspaces,

  function( morphism, test_morphism )
    local kernel_matrix;
    
    kernel_matrix := SyzygiesOfRows( morphism!.morphism );

    return VectorSpaceMorphism( Source( test_morphism ),
                    RightDivide( test_morphism!.morphism, kernel_matrix ),
                    QVectorSpace( NrRows( kernel_matrix ) ) );
    
end );
\end{Verbatim}
\end{small}

There are two things to mention here.
First thing, instead of \texttt{KernelLift} we could also have installed another method,
called \texttt{KernelLiftWithGivenKernelObject}. The result would again be an implementation of \texttt{KernelLift}, but those
\texttt{WithGiven} operations offer an advantage. Instead of calling a function with only $\alpha$ and $\tau$, there
would be a third argument, which should be equal to the output of \texttt{KernelObject} and then serves as the range of the output morphism.
This can be done to ensure better compatibility if the given equality function for objects might not be complete, or to save computation time.
Please have a look at \ref{in_text:with_given} for more details about the \texttt{WithGiven} functions.

Also, the function given for \texttt{KernelLift} does at the beginning the same as the function given for \texttt{KernelEmbedding}. This might lead to
an unnecessary computation. So, in this context, it would be better to provide a more general function for \texttt{LiftAlongMonomorphism}.
The input for this function would be a mono, and some morphism, and it would then provide a lift. The function for \texttt{KernelLift} would then be derived
from this operation and \texttt{KernelEmbedding}, by computing
\[
 \mathrm{LiftAlongMonomorphism} \left( \mathrm{KernelEmbedding} \left( \alpha \right), \tau \right).
\]
For more information about this system, please refer to the derivations section \ref{section:derivations}.
An implementation of the kernel would then instead look like this.

\begin{small}
\begin{Verbatim}[frame=single]
AddKernelEmb( vecspaces,

  function( morphism )
    local kernel_emb, kernel_obj;
    
    kernel_emb := SyzygiesOfRows( morphism!.morphism );
    
    kernel_obj := QVectorSpace( NrRows( kernel_emb ) );
    
    return VectorSpaceMorphism( kernel_obj, kernel_emb, Source( morphism ) );
    
end );

AddMonoAsKernelLift( vecspaces,

  function( monomorphism, test_morphism )

    return VectorSpaceMorphism( Source( test_morphism ),
           RightDivide( test_morphism!.morphism, monomorphism!.morphism ),
           Source( monomorphism ) );

end );
\end{Verbatim}
\end{small}

The code duplication from above disappear.

Now, we continue by adding more functionality to the category. Dual to our second implementation of the kernel,
we will provide functions for the cokernel.

\begin{small}
\begin{Verbatim}[frame=single]
AddCokernelProj( vecspaces,

  function( morphism )
    local cokernel_proj, cokernel_obj;

    cokernel_proj := SyzygiesOfColumns( morphism!.morphism );

    cokernel_obj := QVectorSpace( NrColumns( cokernel_proj ) );

    return VectorSpaceMorphism( Range( morphism ), 
              cokernel_proj, cokernel_obj );

end );

AddColiftAlongEpimorphism( vecspaces,
  
  function( epimorphism, test_morphism )
    
    return VectorSpaceMorphism( Range( epimorphism ),
              LeftDivide( epimorphism!.morphism, test_morphism!.morphism ),
              Range( test_morphism ) );
    
end );
\end{Verbatim}
\end{small}

The next step is implementing functions for the \textrm{ZeroObject} in the category, i.e. the vector space
of dimension 0. A proper implementation here needs three algorithms. One, without any arguments, returns the
zero object of the category. The other two should, given one object, compute the unique morphisms from and into
the zero object. Those are named \texttt{UniversalMorphismIntoZeroObject} and \texttt{UniversalMorphismFromZeroObject}.
For those there are again two choices, either using the \texttt{WithGiven} operations or not.
For our example we do both, providing functions for the \texttt{WithGiven} operations and for the ones that
need to compute their own zero object. Please note that this is again possible because of the installed equality of objects.

\begin{small}
\begin{Verbatim}[frame=single]
AddZeroObject( vecspaces,

  function( )

    return QVectorSpace( 0 );

end );

AddUniversalMorphismIntoZeroObject( vecspaces,

  function( source )

    return VectorSpaceMorphism( source,
              HomalgZeroMatrix( Dimension( source ), 0, VECTORSPACES_FIELD ),
              QVectorSpace( 0 ) );
    
end );

AddUniversalMorphismIntoZeroObjectWithGivenZeroObject( vecspaces,

  function( source, terminal_object )

    return VectorSpaceMorphism( source,
              HomalgZeroMatrix( Dimension( source ), 0, VECTORSPACES_FIELD ),
              terminal_object );
    
end );

AddUniversalMorphismFromZeroObject( vecspaces,

  function( sink )
    
    return VectorSpaceMorphism( QVectorSpace( 0 ),
              HomalgZeroMatrix( 0, Dimension( sink ), VECTORSPACES_FIELD ),
              sink );
    
end );

AddUniversalMorphismFromZeroObjectWithGivenZeroObject( vecspaces,

  function( sink, initial_object )
    
    return VectorSpaceMorphism( initial_object,
                   HomalgZeroMatrix( 0, Dimension( sink ), VECTORSPACES_FIELD ),
                   sink );
    
end );
\end{Verbatim}
\end{small}

Now we turn the set of homomorphisms between two objects into an Abelian group. For this, we need to define the addition of
two morphisms having the same sources and ranges, the additive inverse of a morphism, and the zero morphism between two objects.

\begin{small}
\begin{Verbatim}[frame=single]
AddAdditionForMorphisms( vecspaces,
                         
  function( a, b )
    
    return VectorSpaceMorphism( Source( a ),
              a!.morphism + b!.morphism,
              Range( a ) );
    
end );

AddAdditiveInverseForMorphisms( vecspaces,
                                
  function( a )
    
    return VectorSpaceMorphism( Source( a ),
              - a!.morphism,
              Range( a ) );
    
end );

AddZeroMorphism( vecspaces,
                     
  function( a, b )
    
    return VectorSpaceMorphism( a,
              HomalgZeroMatrix( Dimension( a ),
                                Dimension( b ),
                                VECTORSPACES_FIELD ),
              b );
    
end );
\end{Verbatim}
\end{small}

The installation of \texttt{ZeroMorphism} is not necessary here, since we have the zero object and the corresponding
universal morphisms, this method could also be derived from those. Of course, providing a primitive implementation might
increase the speed of some computations.

The last thing left to implement is the direct sum and its universal properties. After that, we can tell the system that the
category is abelian, finalize it, and start computations. The direct sum needs, for a proper implementation, at least five
functions. At first, the direct sum of objects needs to be created. Then there are injections of the components and projections
to the factors. Finally, since the direct sum is product and coproduct at the same time, there are two universal properties to be implemented.
For details please have a look at Chapter \ref{chapter:constructions}.

\begin{small}
\begin{Verbatim}[frame=single]
AddDirectSum( vecspaces,

  function( object_product_list )
    local dim;
    
    dim := Sum( List( object_product_list, c -> Dimension( c ) ) );
    
    return QVectorSpace( dim );
  
end );

AddInjectionOfCofactorOfDirectSum( vecspaces,

  function( object_product_list, injection_number )
    local components, dim, dim_pre, dim_post, dim_cofactor, coproduct,
    number_of_objects, injection_of_cofactor;
    
    components := object_product_list;
    
    number_of_objects := Length( components );
    
    dim := Sum( components, c -> Dimension( c ) );
    
    dim_pre := Sum( components{ [ 1 .. injection_number - 1 ] },
                    c -> Dimension( c ) );
    
    dim_post := Sum( components{ [ injection_number + 1 .. number_of_objects ] },
                     c -> Dimension( c ) );
    
    dim_cofactor := Dimension( object_product_list[ injection_number ] );
    
    coproduct := QVectorSpace( dim );
    
    injection_of_cofactor := HomalgZeroMatrix( dim_cofactor,
                                               dim_pre,
                                               VECTORSPACES_FIELD );
    
    injection_of_cofactor := UnionOfColumns( injection_of_cofactor, 
                                         HomalgIdentityMatrix( dim_cofactor,
                                            VECTORSPACES_FIELD ) );
    
    injection_of_cofactor := UnionOfColumns( injection_of_cofactor, 
                                    HomalgZeroMatrix( dim_cofactor,
                                                      dim_post,
                                                      VECTORSPACES_FIELD ) );
    
    return VectorSpaceMorphism( object_product_list[ injection_number ],
                                injection_of_cofactor, coproduct );

end );

AddUniversalMorphismFromDirectSum( vecspaces,

  function( diagram, sink )
    local dim, coproduct, components, universal_morphism, morphism;
    
    components := sink;
    
    dim := Sum( components, c -> Dimension( Source( c ) ) );
    
    coproduct := QVectorSpace( dim );
    
    universal_morphism := sink[1]!.morphism;
    
    for morphism in components{ [ 2 .. Length( components ) ] } do
    
      universal_morphism := UnionOfRows( universal_morphism,
                                         morphism!.morphism );
  
    od;
  
    return VectorSpaceMorphism( coproduct, universal_morphism,
                                   Range( sink[1] ) );
  
end );

AddProjectionInFactorOfDirectSum( vecspaces,

  function( object_product_list, projection_number )
    local components, dim, dim_pre, dim_post, dim_factor, direct_product,
          number_of_objects, projection_in_factor;
    
    components := object_product_list;
    
    number_of_objects := Length( components );
    
    dim := Sum( components, c -> Dimension( c ) );
    
    dim_pre := Sum( components{ [ 1 .. projection_number - 1 ] },
                    c -> Dimension( c ) );
    
    dim_post := Sum( components{ [ projection_number + 1 
                                   .. number_of_objects ] },
                                   c -> Dimension( c ) );
    
    dim_factor := Dimension( object_product_list[ projection_number ] );
    
    direct_product := QVectorSpace( dim );
    
    projection_in_factor := HomalgZeroMatrix( dim_pre,
                                              dim_factor,
                                              VECTORSPACES_FIELD );
    
    projection_in_factor := UnionOfRows( projection_in_factor, 
                                         HomalgIdentityMatrix( dim_factor,
                                            VECTORSPACES_FIELD ) );
    
    projection_in_factor := UnionOfRows( projection_in_factor, 
                                         HomalgZeroMatrix( dim_post,
                                                    dim_factor,
                                                    VECTORSPACES_FIELD ) );
    
    return VectorSpaceMorphism( direct_product,
                                projection_in_factor,
                                object_product_list[ projection_number ] );

end );

AddUniversalMorphismIntoDirectSum( vecspaces,

  function( diagram, sink )
    local dim, direct_product, components, universal_morphism, morphism;
    
    components := sink;
    
    dim := Sum( components, c -> Dimension( Range( c ) ) );
    
    direct_product := QVectorSpace( dim );
    
    universal_morphism := sink[1]!.morphism;
    
    for morphism in components{ [ 2 .. Length( components ) ] } do
    
      universal_morphism := UnionOfColumns( universal_morphism,
                                            morphism!.morphism );
  
    od;
  
    return VectorSpaceMorphism( Source( sink[1] ),
                                universal_morphism,
                                direct_product );
  
end );
\end{Verbatim}
\end{small}

Now we can finalize the category.

\begin{small}
\begin{Verbatim}[frame=single]
Finalize( vecspaces );
\end{Verbatim}
\end{small}

The finalize step is very important. It triggers those derivations of methods whose correctness
rely on the fact that the user won't provide further methods for basic operations in the future.
Also, it makes the call of category constructors on $\Svec$ possible, e.g.,
one could create the category of so-called generalized morphisms over $\Svec$.

We are ready to perform some computations with our implementation of $\Svec$.
In order to have a nice output in the \GAP shell, we also install view methods.

\begin{small}
\begin{Verbatim}[frame=single]
##
InstallMethod( ViewObj,
               [ IsSQVecObj ],

  function( a )

    Print( "<An object in SQVec of dimension ",
           String( Dimension( a ) ), ">" );

end );

##
InstallMethod( ViewObj,
               [ IsSQVecMor ],

  function( morphism )

    Print( "A morphism in SQVec with underlying matrix:\n" );
  
    Display( UnderlyingMatrix( morphism ) );

end );
\end{Verbatim}
\end{small}

\subsection{Computation}\label{subsection:skeletal_vec_computation}

We initialize an interactive session with our implementation of the category of skeletal rational vector spaces $\Svec$
and start by creating some morphisms.

\begin{small}
\begin{Verbatim}[commandchars=!@\%,frame=single]
!color@blue%@gap>%!color@red%@ V := QVectorSpace( 2 );%
<A rational vector space of dimension 2>
!color@blue%@gap>%!color@red%@ W := QVectorSpace( 3 );%
<A rational vector space of dimension 3>
!color@blue%@gap>%!color@red%@ alpha := VectorSpaceMorphism( V, [ [ 1, 1, 1 ], [ -1, -1, -1 ] ], W );%
A rational vector space homomorphism with matrix: 
[ [   1,   1,   1 ],
  [  -1,  -1,  -1 ] ]
\end{Verbatim}

\end{small}

Unsurprisingly, we can now compute some of the stuff we told the system how to compute, for example the \texttt{KernelEmbedding},
which is the embedding of the kernel. Also, the cokernel and its projection can be computed. Those computations will
be carried out using the functions we have given to the system. It is also possible to use the standard arithmetic on the morphism.
Even if the added functions have different names, e.g., \texttt{AdditionForMorphisms}, it is possible to use the \GAP arithmetic
operations, e.g., \texttt{+}, for it.

\begin{small}
\begin{Verbatim}[commandchars=!@\%,frame=single]
!color@blue%@gap>%!color@red%@ KernelEmb( alpha );%
A rational vector space homomorphism with matrix:
[ [  1,  1 ] ]
!color@blue%@gap>%!color@red%@ Cokernel( alpha );%
<A rational vector space of dimension 2>
!color@blue%@gap>%!color@red%@ CokernelProj( alpha );%
A rational vector space homomorphism with matrix: 
[ [  -1,  -1 ],
  [   1,   0 ],
  [   0,   1 ] ]
!color@blue%@gap>%!color@red%@ alpha + alpha;%
A rational vector space homomorphism with matrix: 
[ [   2,   2,   2 ],
  [  -2,  -2,  -2 ] ]
!color@blue%@gap>%!color@red%@ - alpha;%
A rational vector space homomorphism with matrix: 
[ [  -1,  -1,  -1 ],
  [   1,   1,   1 ] ]
\end{Verbatim}

\end{small}

The fact that such functions work properly is not surprising at all. But \CapPkg has the power to derive
further constructions from the given functions. Here are some examples of methods now applicable.

\begin{small}
\begin{Verbatim}[commandchars=!@\%,frame=single]
!color@blue%@gap>%!color@red%@ IsMonomorphism( alpha );%
false
!color@blue%@gap>%!color@red%@ IsEpimorphism( alpha );%
false
!color@blue%@gap>%!color@red%@ alpha_image := ImageEmbedding( alpha );%
A rational vector space homomorphism with matrix: 
[ [  1,  1,  1 ] ]
\end{Verbatim}

\end{small}

This \texttt{ImageEmbedding} is probably\footnote{The list of derivations in \CapPkg is ever-growing.} computed by taking the \texttt{KernelEmbedding} of the \textrm{CokernelProjection}.
There are many more constructions computable right now. One can see a list of all possible operations in the category
in the category using the 
\begin{itemize}
  \item \texttt{InstalledMethodsOfCategory}
\end{itemize}
command.

\begin{small}
\begin{Verbatim}[commandchars=!@\%,frame=single]
!color@blue%@gap>%!color@red%@ InstalledMethodsOfCategory( vecspaces );%
Can do the following basic methods at the moment:
* InverseImmutable, weight 201
... <output ommitted> ..
\end{Verbatim}

\end{small}

Now, we can also carry out less obvious computations. In the next example we compute the intersection of two 2 dimensional
subspaces of a 3 dimensional vector space.
We model subobjects of an object $V \in \Svec$ by monomorphisms with range $V$.
Taking the intersection of two subobjects 
$\alpha: W \hookrightarrow V$ and $\beta: U \hookrightarrow V$
corresponds to taking the fiber product of $\alpha$ and $\beta$ and return the natural morphism from this fiber product to $V$.
Please note that this point of view is completely general in the sense that we may always think of intersections of
subobjects as fiber product in all contexts where fiber products exists, for example for ideals or submodules.

\begin{small}
\begin{Verbatim}[commandchars=!@\%,frame=single]
!color@blue%@gap>%!color@red%@ alpha := VectorSpaceMorphism( V, [ [ 1, 0, 0 ], [ 0, 1, 1 ] ], W );%
A rational vector space homomorphism with matrix: 
[ [  1,  0,  0 ],
  [  0,  1,  1 ] ]
!color@blue%@gap>%!color@red%@ beta := VectorSpaceMorphism( V, [ [ 1, 1, 0 ], [ 0, 0, 1 ] ], W );%
A rational vector space homomorphism with matrix: 
[ [  1,  1,  0 ],
  [  0,  0,  1 ] ]
!color@blue%@gap>%!color@red%@ fiberproduct := FiberProduct( alpha, beta );%
<A rational vector space of dimension 1>
!color@blue%@gap>%!color@red%@ projection := ProjectionInFactor( fiberproduct, 1 );%
A rational vector space homomorphism with matrix: 
[ [  1,  1 ] ]
!color@blue%@gap>%!color@red%@ intersection := PreCompose( projection, alpha );%
A rational vector space homomorphism with matrix: 
[ [  1,  1,  1 ] ]
\end{Verbatim}

\end{small}

But the computational powers of \CapPkg do not end at this point.
From the basic operations provided by \CapPkg, we may, for example,
implement a function for the computation of the famous snake lemma morphism as follows:
\begin{small}
  \begin{small}
\begin{Verbatim}[frame=single]
snake_lemma_morphism := function( delta, beta, lambda )
    local gamma, iota, epsilon, mu, alpha, pi, p1, p2, q1, q2, u;
    
    gamma := CokernelColift( delta, PreCompose( beta, lambda ) );
    iota := KernelEmbedding( gamma );
    epsilon := CokernelProjection( delta );
    mu := KernelEmbedding( lambda );
    alpha := KernelLift( lambda, PreCompose( delta, beta ) );
    pi := CokernelProjection( alpha );
    
    p1 := ProjectionInFactorOfFiberProduct( [ iota, epsilon ], 1 );
    p2 := ProjectionInFactorOfFiberProduct( [ iota, epsilon ], 2 );
    q1 := InjectionOfCofactorOfPushout( [ mu, pi ], 1 );
    q2 := InjectionOfCofactorOfPushout( [ mu, pi ], 2 );
    u := ColiftAlongEpimorphism( p1, PreCompose( [ p2, beta, q1 ] ) );

    return LiftAlongMonomorphism( q2, u );
    
end;
\end{Verbatim}
\end{small}

\end{small}
Actually, this is a translation of a proof of the snake lemma given in \cite{MLCWM} to \CapPkg.
We will try it on a concrete example:

\begin{center}
  \begin{tikzpicture}[label/.style={postaction={
    decorate,
    decoration={markings, mark=at position .5 with \node #1;}}}]
    \coordinate (r) at (3.8,0);
    \coordinate (d) at (0,-2.4);
    
    \node (A1) {};
    \node (B1) at ($(A1)+0.5*(r)$) {${\Q}^2$};
    \node (C1) at ($(B1)+(r)$) {${\Q}^2$};
    \node (D1) at ($(C1)+(r)$) {$\cokernel( \delta )$};
    \node (E1) at ($(D1)+0.5*(r)$) {$0$};
    
    \node (A2) at ($(A1)+(d)$) {$0$};
    \node (B2) at ($(A2)+0.5*(r)$) {$\kernel( \lambda )$};
    \node (C2) at ($(B2)+(r)$) {${\Q}^3$};
    \node (D2) at ($(C2)+(r)$) {${\Q}^1$};
    \node (E2) at ($(D2)+0.5*(r)$) {};
    
    \node (D0) at ($(D1)-(d)$) {$\kernel( \gamma )$};
    
    \node (B3) at ($(B2)+(d)$) {$\cokernel( \alpha )$};
    
    \draw[->,thick] (B1) --node[above]{
      \color{blue}
      $
      \delta :=
        \begin{pmatrix}
            1 & 0 \\
            0 & 0
        \end{pmatrix}
      $
    } (C1);
    \draw[->,thick] (C1) --node[above]{
      $\epsilon$
    } (D1);
    \draw[->,thick] (D1) -- (E1);
    
    \draw[->,thick] (A2) -- (B2);
    \draw[->,thick] (B2) --node[below]{
      $\mu$
    } (C2);
    \draw[->,thick] (C2) --node[below]{
      \color{blue}
      $
      \lambda :=
        \begin{pmatrix}
            0 \\
            0 \\
            1
        \end{pmatrix}
      $
    } (D2);
    
    \draw[->,thick] (D0) --node[right]{
      $\iota$
    } (D1);
    \draw[->,thick] (B2) --node[left]{
      $\pi$
    } (B3);
    
    \draw[->,thick] (B1) --node[left]{
      $\alpha$
    } (B2);
    \draw[->,thick] (C1) --node[right]{
      \color{blue}
      $
      \beta :=
        \begin{pmatrix}
            2 & 4 & 0 \\
            3 & 5 & 0
        \end{pmatrix}
      $
    } (C2);
    \draw[->,thick] (D1) --node[right]{
      $\gamma$
    } (D2);
    
  \end{tikzpicture}
\end{center}

\begin{small}
\begin{Verbatim}[commandchars=!@\%,frame=single]
!color@blue%@gap>%!color@red%@ LoadPackage( "HomologicalAlgebraForCAP" );%
true
!color@blue%@gap>%!color@red%@ V1 := QVectorSpace( 1 );%
<A rational vector space of dimension 1>
!color@blue%@gap>%!color@red%@ V2 := QVectorSpace( 2 );%
<A rational vector space of dimension 2>
!color@blue%@gap>%!color@red%@ V3 := QVectorSpace( 3 );%
<A rational vector space of dimension 3>
!color@blue%@gap>%!color@red%@ alpha2 := VectorSpaceMorphism( V3, [ [ 0, 0 ], [ 1, 0 ], [ 0, 1 ] ], V2 );%
A rational vector space homomorphism with matrix: 
[ [  0,  0 ],
  [  1,  0 ],
  [  0,  1 ] ]
!color@blue%@gap>%!color@red%@ beta1 := VectorSpaceMorphism( V2, [ [ 1, 0, 0 ], [ 0, 1, 0 ] ], V3 );%
A rational vector space homomorphism with matrix: 
[ [  1,  0,  0 ],
  [  0,  1,  0 ] ]
!color@blue%@gap>%!color@red%@ gamma1 := VectorSpaceMorphism( V1, [ [ 1, 0 ] ], V2 );%
A rational vector space homomorphism with matrix: 
[ [  1,  0 ] ]
!color@blue%@gap>%!color@red%@ gamma2 := IdentityMorphism( V3 );%
A rational vector space homomorphism with matrix: 
[ [  1,  0,  0 ],
  [  0,  1,  0 ],
  [  0,  0,  1 ] ]
!color@blue%@gap>%!color@red%@ gamma3 := VectorSpaceMorphism( V2, [ [ 0 ], [ 1 ] ], V1 );%
A rational vector space homomorphism with matrix: 
[ [  0 ],
  [  1 ] ]
!color@blue%@gap>%!color@red%@ snake := SnakeLemmaConnectingHomomorphism( alpha2, gamma1, gamma2,%
!color@blue%@>%!color@red%@ gamma3, beta1 );%
A rational vector space homomorphism with matrix: 
[ [  1 ] ]
\end{Verbatim}
 
\end{small}
\emph{Challenge}: how does the snake morphism look like if we computed with the same matrices, but interpreted 
as morphisms between free abelian groups of appropriate finite ranks?

Another nice feature of \CapPkg are functors. The functors in \CapPkg should represent mathematical functors between categories and are modeled
as morphisms in the \CapPkg category of categories, \texttt{CapCat}. They provide a good way to communicate between different data structures implemented
by someone in the \CapPkg environment. Then they are functors between different categories. Also, it is useful to implement some internal operations
in a category as endofunctors, since functors can be manipulated like morphisms, i.e. composed, etc. Also, natural transformations can be implemented in
\CapPkg. Generally, functors in \CapPkg need two functions, a function describing the action on the objects, and a function describing the actions on the
morphism. Also, we need to tell the system in which component the functor is covariant and in which the functor is contravariant. Covariant is hereby the standard
case, so we do not need to specify it.
In the next example, we create, as a first example, the identity functor of the previously created category of skeletal vector spaces $\Svec$. This is somehow the easiest functor,
since it does not require any manipulation at all. We start by creating the functor object, and then adding the object and the morphism function to it.

\begin{small}
\begin{Verbatim}[commandchars=!@\%,frame=single]
!color@blue%@gap>%!color@red%@ id_functor := CapFunctor( "Identity of vecspaces", vecspaces, vecspaces );%
Identity of vecspaces
!color@blue%@gap>%!color@red%@ AddObjectFunction( id_functor, IdFunc );%
!color@blue%@gap>%!color@red%@ AddMorphismFunction( id_functor, function( obj1, mor, obj2 )%
!color@blue%@>%!color@red%@ return mor; end );%
\end{Verbatim}
 
\end{small}

Even if this example is trivial, we already see one of the main points to keep in mind when creating functors. Although the functor only takes one argument,
the morphism function takes three. This is due to the fact that morphism functions in functors are always kind of \texttt{WithGiven} functions
(see \ref{in_text:with_given}). The first
argument here is the image of the source(s) of the morphism(s) given to the object function of the functor, the last is the image of the range(s).

It was already mentioned that functors are morphisms in a specific category, and every \CapPkg category has a specific object in this category.
Of course, a method for the identity morphism is implemented in \texttt{CapCat}, so we can have the same result with less effort.

\begin{small}
\begin{Verbatim}[commandchars=!@\%,frame=single]
!color@blue%@gap>%!color@red%@ id_functor := IdentityMorphism( AsCatObject( vecspaces ) );%
Identity functor of VectorSpaces
\end{Verbatim}

\end{small}

The identity functor is not of much computational use, but interesting to show how a functor is created in general. We will now create a second functor. It will
map a vector space $V$ to its direct sum with itself, i.e. $V \oplus V$, and provides the diagonal of a morphism. The functor is created as follows.

\begin{small}
\begin{Verbatim}[frame=single]
double_functor := CapFunctor( "DoubleOfVecspaces", 
                              vecspaces, vecspaces );

AddObjectFunction( double_functor,
                   
  function( obj )
    
    return QVectorSpace( 2 * Dimension( obj ) );
    
end );

AddMorphismFunction( double_functor,
                     
  function( new_source, mor, new_range )
    local matr, matr1;
    
    matr := EntriesOfHomalgMatrixAsListList( mor!.morphism );
    
    matr := Concatenation( List( matr,
            i -> Concatenation( i,
                 ListWithIdenticalEntries( Length( i ), 0 ) ) ),
                           List( matr,
            i -> Concatenation(
                 ListWithIdenticalEntries( Length( i ), 0 ), i ) ) );
    
    return VectorSpaceMorphism( new_source, matr, new_range );
    
end );
\end{Verbatim}
\end{small}

Since we have given this functor functions about how to act on objects or morphisms, we can actually apply it to such. We can also compose it with itself
and then again apply it to objects or morphisms.

\begin{small}
\begin{Verbatim}[commandchars=!@\%,frame=single]
!color@blue%@gap>%!color@red%@ V2;%
<A rational vector space of dimension 2>
!color@blue%@gap>%!color@red%@ ApplyFunctor( double_functor, V2 );%
<A rational vector space of dimension 4>
!color@blue%@gap>%!color@red%@ alpha2;%
A rational vector space homomorphism with matrix:
[ [  0,  0 ],
  [  1,  0 ],
  [  0,  1 ] ]
!color@blue%@gap>%!color@red%@ ApplyFunctor( double_functor, alpha2 );%
A rational vector space homomorphism with matrix:
[ [  0,  0,  0,  0 ],
  [  1,  0,  0,  0 ],
  [  0,  1,  0,  0 ],
  [  0,  0,  0,  0 ],
  [  0,  0,  1,  0 ],
  [  0,  0,  0,  1 ] ]
\end{Verbatim}

\end{small}

As said, since functors are morphisms, it is also possible to manipulate functors like morphisms. In the next example, we compose the double functor with itself
to get a quadruble functor.

\begin{small}
\begin{Verbatim}[commandchars=!@\%,frame=single]
!color@blue%@gap>%!color@red%@ quadruple_functor := PreCompose( double_functor, double_functor );%
Composition of DoubleOfVecspaces and DoubleOfVecspaces
!color@blue%@gap>%!color@red%@ ApplyFunctor( quadruple_functor, V2 );%
<A rational vector space of dimension 8>
!color@blue%@gap>%!color@red%@ ApplyFunctor( quadruple_functor, alpha2 );%
A rational vector space homomorphism with matrix: 
[ [  0,  0,  0,  0,  0,  0,  0,  0 ],
  [  1,  0,  0,  0,  0,  0,  0,  0 ],
  [  0,  1,  0,  0,  0,  0,  0,  0 ],
  [  0,  0,  0,  0,  0,  0,  0,  0 ],
  [  0,  0,  1,  0,  0,  0,  0,  0 ],
  [  0,  0,  0,  1,  0,  0,  0,  0 ],
  [  0,  0,  0,  0,  0,  0,  0,  0 ],
  [  0,  0,  0,  0,  1,  0,  0,  0 ],
  [  0,  0,  0,  0,  0,  1,  0,  0 ],
  [  0,  0,  0,  0,  0,  0,  0,  0 ],
  [  0,  0,  0,  0,  0,  0,  1,  0 ],
  [  0,  0,  0,  0,  0,  0,  0,  1 ] ]
\end{Verbatim}

\end{small}

Some functors are more important for applications than others. If some data structure is implemented in the \CapPkg setup, and if some functors are implemented
in this setup, one might want them to work like methods, and does not want to use \texttt{ApplyFunctor} all the time. This is possible using the \texttt{InstallFunctor}
command. The example also shows an interesting fact about the caching. Both the functor used with \texttt{ApplyFunctor} and the installed version use the same cache. This
gives a bit more consistency and might save computation time. Also, it ensures the interchangeability of both ways of applying the functor.

\begin{small}
\begin{Verbatim}[commandchars=!@\%,frame=single]
!color@blue%@gap>%!color@red%@ InstallFunctor( double_functor, "DoubleFunctor" );%
!color@blue%@gap>%!color@red%@ V4 := DoubleFunctor( V2 );%
<A rational vector space of dimension 4>
!color@blue%@gap>%!color@red%@ V4_2 := ApplyFunctor( double_functor, V2 );%
<A rational vector space of dimension 4>
!color@blue%@gap>%!color@red%@ IsIdenticalObj( V4, V4_2 );%
true
!color@blue%@gap>%!color@red%@ DoubleFunctor( alpha2 );%
A rational vector space homomorphism with matrix: 
[ [  0,  0,  0,  0 ],
  [  1,  0,  0,  0 ],
  [  0,  1,  0,  0 ],
  [  0,  0,  0,  0 ],
  [  0,  0,  1,  0 ],
  [  0,  0,  0,  1 ] ]
\end{Verbatim}

\end{small}

At last, we want to show natural transformations in \CapPkg and the manipulation which is possible for them. For this, we create the natural transformation which
swaps the components in the double functor, i.e., for some vector space $V$ it creates the morphism $V \oplus V \rightarrow V \oplus V$, $\left( x,y \right) \mapsto \left( y,x \right)$.
A natural transformation only needs one function, which returns for an object the correct morphism.

\begin{small}
\begin{Verbatim}[frame=single]
double_swap_components := NaturalTransformation( "double swap components", 
                            double_functor, double_functor );

AddNaturalTransformationFunction( double_swap_components,
  
  function( doubled_source, obj, doubled_range )
    local zero_morphism, one_morphism;
    
    zero_morphism := ZeroMorphism( obj, obj );
    
    one_morphism := IdentityMorphism( obj );
    
    return MorphismBetweenDirectSums( [ [ zero_morphism, one_morphism ],
                                        [ one_morphism, zero_morphism ] ] );
    
end );
\end{Verbatim}
\end{small}

Of course, we can apply the natural transformation to objects. But it is also possible to compute the vertical and the horizontal composition of natural transformations.
In the next and last example we compute both the horizontal and the vertical composition of the natural transformation we created with itself. We leave it to the
reader to verify the results.

\begin{small}
\begin{Verbatim}[commandchars=!@\%,frame=single]
!color@blue%@gap>%!color@red%@ ApplyNaturalTransformation( double_swap_components, V2 );%
A rational vector space homomorphism with matrix: 
[ [  0,  0,  1,  0 ],
  [  0,  0,  0,  1 ],
  [  1,  0,  0,  0 ],
  [  0,  1,  0,  0 ] ]
!color@blue%@gap>%!color@red%@ h_composition := HorizontalPreCompose( double_swap_components,%
                                            double_swap_components );
Vertical composition of Horizontal composition of natural
transformation double swap components and functor DoubleOfVecspaces
and Horizontal composition of functor DoubleOfVecspaces
and natural transformation double swap components
!color@blue%@gap>%!color@red%@ ApplyNaturalTransformation( h_composition, V2 );%
A rational vector space homomorphism with matrix: 
[ [  0,  0,  0,  0,  0,  0,  1,  0 ],
  [  0,  0,  0,  0,  0,  0,  0,  1 ],
  [  0,  0,  0,  0,  1,  0,  0,  0 ],
  [  0,  0,  0,  0,  0,  1,  0,  0 ],
  [  0,  0,  1,  0,  0,  0,  0,  0 ],
  [  0,  0,  0,  1,  0,  0,  0,  0 ],
  [  1,  0,  0,  0,  0,  0,  0,  0 ],
  [  0,  1,  0,  0,  0,  0,  0,  0 ] ]
!color@blue%@gap>%!color@red%@ v_composition := VerticalPreCompose( double_swap_components,%
                                          double_swap_components );
Vertical composition of double swap components and double swa
 p components
!color@blue%@gap>%!color@red%@ ApplyNaturalTransformation( v_composition, V2 );%
A rational vector space homomorphism with matrix: 
[ [  1,  0,  0,  0 ],
  [  0,  1,  0,  0 ],
  [  0,  0,  1,  0 ],
  [  0,  0,  0,  1 ] ]
\end{Verbatim}

\end{small}
