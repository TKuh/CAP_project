\section{Categories}

Classically, a category consists of a class of objects, a set of morphisms, identity morphisms, and a composition function
satisfying some simple axioms. In $\CapPkg$, we use a slightly different notion of a category.

\begin{definition}
 A \CapPkg category $\mathbf{C}$ consists of the following data:
 \begin{enumerate}
  \item A set $\Obj_{\mathbf{C}}$ of \textbf{objects}.
  \item For every pair $a,b \in \Obj_{\mathbf{C}}$, a set $\Hom_{\mathbf{C}}( a, b )$ of \textbf{morphisms}.
  \item For every pair $a,b \in \Obj_{\mathbf{C}}$, an equivalence relation $\sim_{a,b}$ on $\Hom_{\mathbf{C}}( a, b )$
  called \textbf{congruence for morphisms}.
  \item For every $a \in \Obj_{\mathbf{C}}$, an \textbf{identity morphism} $\id_a \in \Hom_{\mathbf{C}}( a, a )$.
  \item For every triple $a, b, c \in \Obj_{\mathbf{C}}$, a \textbf{composition function}
  \[
   \circ: \Hom_{\mathbf{C}}( b, c ) \times \Hom_{\mathbf{C}}( a, b ) \rightarrow \Hom_{\mathbf{C}}( a, c )
  \]
  compatible with the congruence, i.e., 
  if $\alpha, \alpha' \in \Hom_{C}( a, b )$, 
  $\beta, \beta' \in \Hom_{C}( b, c )$,
  $\alpha \sim_{a,b} \alpha'$ 
  and $\beta \sim_{b,c} \beta'$, 
  then $\beta \circ \alpha \sim_{a,c} \beta' \circ \alpha'$.
  \item For all $a, b \in \Obj_{\mathbf{C}}$, 
        $\alpha \in \Hom_{\mathbf{C}}( a, b )$, 
        we have 
        \[
        \left( \id_{b} \circ \alpha \right) \sim_{a,b} \alpha
        \]
        and
        \[
        \alpha \sim_{a,b} \left( \alpha \circ \id_{a} \right).
        \]
  \item For all $a,b,c,d \in \Obj_{\mathbf{C}}$, 
        $\alpha \in \Hom_{\mathbf{C}}( a, b )$, 
        $\beta \in \Hom_{\mathbf{C}}( b, c )$, 
        $\gamma \in \Hom_{\mathbf{C}}( c, d )$,
        we have
        \[
        \left(( \gamma \circ \beta ) \circ \alpha \right) \sim_{a,d} \left( \gamma \circ ( \beta \circ \alpha ) \right)
        \]
 \end{enumerate}
\end{definition}

So the main differences between a $\CapPkg$ category and a classical category are:
\begin{enumerate}
 \item A $\CapPkg$ category has a set of objects, not a class.
 \item A $\CapPkg$ cateogry has as an additional part of its datum a congruence for morphisms, and the
 axioms are stated with respect to this congruence, and not with respect to equality.
\end{enumerate}

We will see that the congruence for morphisms actually makes the implementation of some categories easier (for example
the category of presentations).

\begin{remark}
 Passing to the quotient sets $\Hom_{C}( a, b )/\sim_{a,b}$ gives rise to a classical category $\mathbf{D}$, because
 all constructions and axioms respects the congruence.
 It is usually the case that we actually want to study $\mathbf{D}$, but that it is easier to implement a $\CapPkg$ category $\mathbf{C}$
 giving rise to $\mathbf{D}$.
\end{remark}

\begin{remark}
 In terms of higher category theory, a $\CapPkg$ category is a $2$-category such that the $2$-morphism sets are either empty or a singleton.
\end{remark}

\begin{convention}
 Throughout this manual we will use \textit{category} as a short term for a \CapPkg category. 
 If we want to refer to the classical notion of a category (e.g. used in \cite{MLCWM}) we will use the term \textit{classical category}.
\end{convention}


\section{Equality Specifications}

\begin{specification}
 Every basic operation has to yield equal output for given equal input.
\end{specification}

\subsection{\GAP Sets and \GAP Maps}

\section{Typing Specifications}

\begin{specification}
 Every basic operation has to match its type.
\end{specification}

\section{Mathematical Specifications}

\begin{specification}
 Every basic operation has to compute what it is supposed to compute.
\end{specification}